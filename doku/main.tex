% IPA Dokumentation
% Haupdatei
%
% 2013 by Lukas Grimm <ombre@ombre.ch>
\documentclass[a4paper,11pt,oneside]{report}
\usepackage[utf8]{inputenc}
\usepackage[ngerman]{babel}
\usepackage{makeidx}
\usepackage{hyperref}
\usepackage{tabularx}
\usepackage{graphicx}
\usepackage{geometry}
\usepackage{pdflscape}
\usepackage{fancyhdr}
\setlength{\headheight}{15.2pt}
\usepackage{listings}
\usepackage{color}
%\usepackage{menukeys}
\hypersetup{%
  colorlinks=false,% hyperlinks will be black
  pdfborderstyle={/S/U/W 0}% border style will be underline of width 1pt
}
\definecolor{mygreen}{rgb}{0,0.6,0}
\definecolor{mygray}{rgb}{0.5,0.5,0.5}
\definecolor{mymauve}{rgb}{0.58,0,0.82}
\lstloadlanguages{Ruby}
\lstset{ %
  backgroundcolor=\color{white},   % choose the background color; you must add \usepackage{color} or \usepackage{xcolor}
  basicstyle=\footnotesize,        % the size of the fonts that are used for the code
  breakatwhitespace=false,         % sets if automatic breaks should only happen at whitespace
  breaklines=true,                 % sets automatic line breaking
  captionpos=b,                    % sets the caption-position to bottom
  commentstyle=\color{mygreen},    % comment style
  deletekeywords={...},            % if you want to delete keywords from the given language
  escapeinside={\%*}{*)},          % if you want to add LaTeX within your code
  extendedchars=true,              % lets you use non-ASCII characters; for 8-bits encodings only, does not work with UTF-8
  frame=single,                    % adds a frame around the code
  keepspaces=true,                 % keeps spaces in text, useful for keeping indentation of code (possibly needs columns=flexible)
  keywordstyle=\color{blue},       % keyword style
  language=Ruby,                 % the language of the code
  morekeywords={*,...},            % if you want to add more keywords to the set
  numbers=left,                    % where to put the line-numbers; possible values are (none, left, right)
  numbersep=5pt,                   % how far the line-numbers are from the code
  numberstyle=\tiny\color{mygray}, % the style that is used for the line-numbers
  rulecolor=\color{mygray},         % if not set, the frame-color may be changed on line-breaks within not-black text (e.g. comments (green here))
  showspaces=false,                % show spaces everywhere adding particular underscores; it overrides 'showstringspaces'
  showstringspaces=false,          % underline spaces within strings only
  showtabs=false,                  % show tabs within strings adding particular underscores
  stepnumber=1,                    % the step between two line-numbers. If it's 1, each line will be numbered
  stringstyle=\color{mymauve},     % string literal style
  tabsize=2,                       % sets default tabsize to 2 spaces
  title=\lstname                   % show the filename of files included with \lstinputlisting; also try caption instead of title
}
\lstset{literate=%
{Ö}{{\"O}}1
{Ä}{{\"A}}1
{Ü}{{\"U}}1
{ß}{{\ss}}2
{ü}{{\"u}}1
{ä}{{\"a}}1
{ö}{{\"o}}1
}
%\usepackage{epigraph}
%\setlength\epigraphwidth{8cm}
%\setlength\epigraphrule{0pt}

% Definition of columtypes for nice tables
\newcolumntype{b}{X}
\newcolumntype{s}{>{\hsize=.25\hsize}X}
%\newcommand{\heading}[1]{\multicolumn{1}{c}{#1}}
\setlength{\headheight}{13.6pt}
\pagestyle{fancyplain}
\fancyhead[L]{Vorname Nachname}
\fancyhead[C]{\today}
\fancyhead[R]{Firma}
%\fancyfoot[C]{}

\begin{document}
\author{Vorname Nachname \\ 
    Firma \\ 
    \texttt{email@firma.ch} }
\date{\today}
\title{IPA Dokumentation: \\
    Titel }
%\pagestyle{myheadings}
%\markright{Vorname Nachname\hfill \today \hfill}
\maketitle
\section{Dokumentinformationen}
\nopagebreak
\subsection{Änderungskontrolle, Prüfung, Genehmigung}
\begin{tabular}{l | l | l | l}
\textbf{Version} & \textbf{Datum} & \textbf{Name} & \textbf{Beschreibung} \\
Vorlage & 2014-01-01 & Lukas Grimm & Dokumentenvorlage V1.0 \\
Vorlage & 2014-02-07 & Lukas Grimm & Dokumentenvorlage V1.9 \\
X.1 & 2014-02-11 & Lukas Grimm & Informationen von pkorg.ch in Teil I einfliessen / Zeitplan \\
X.2 & 2014-02-22 & Lukas Grimm & Testdruck \\
X.3 & 2014-02-22 & Lukas Grimm & Testdruck nach Latex Anpassungen \\
X.4 & 2014-02-25 & Lukas Grimm & Testdruck II \\
X.5 & 2014-02-25 & Lukas Grimm & Finale Version \\
\end{tabular}

\subsection{Referenzierte Dokumente}
\begin{itemize}
\item
\item 
\end{itemize}




\subsection{Verwendete Abkürzungen}
\begin{tabular}{l | l}
\textbf{Abkürzung} & \textbf{Bedeutung} \\
IPA & Individuelle praktische Arbeit \\
\end{tabular}

\tableofcontents
\listoffigures
\listoftables

\renewcommand{\abstractname}{Management Summary}
\begin{abstract}
\thispagestyle{plain}
\paragraph{Ausgangssituation}

\paragraph{Umsetzung}

\paragraph{Ergebnis}

\end{abstract}
\part{Ablauf und Umfeld}


\chapter{Aufgabenstellung}
Die Punkte 1.1 bis 1.3 sind wortgetreu von der Eingabe des Fachvorgesetzten auf pkorg.ch kopiert worden.
\section{Ausgangslage}
 
\section{Auftragsformulierung}
  
\section{Mittel und Methoden}
\begin{itemize}
	\item 
\end{itemize}
\section{Projektorganisation}
\textbf{Projektausschuss:} \\
Fachvorgesetzter: \\
Hauptexperte:  \\
Zweitexperte:  \\
\textbf{Customer:}  \\
\textbf{Product Owner:} \\
\textbf{Scrum Master:} \\
\textbf{Entwicklungsteam:} \\
\pagebreak
\section{Projektrollen}
Die verschiedenen Projektrollen nach Scrum: \\
\begin{table}[h]
\begin{tabularx}{\textwidth}{ |X|X| }
\hline
Customer & Auftraggeber, dem das Produkt nach Abschluss zur Verfügung gestellt wird \\ \hline
Product Owner & Stellt die strategischen Anforderungen. Festlegung der Ziele, Abnahme der Leistung.\\ \hline
Scrum Master & Verantwortlich für das Gelingen von Scrum. Kümmert sich um Hindernisse und sorgt für ein angenehmes Arbeitsklima. \\ \hline
Entwicklungsteam & Für die Lieferung der Produktfunktionalitäten zuständig. \\ \hline
\end{tabularx}
\caption{Projektrollenbeschrieb}
\end{table}

\chapter{Vorkenntnisse}
\begin{itemize}
	\item Vorkenntnis 1
\end{itemize}  
%\input{part1/vorkenntnisse}

\chapter{Vorarbeiten}
\begin{itemize}
	\item Bestellung Hardware
	\item Setup 
	\item Backuptest
	\item Dokumentation Vorlage in \LaTeX{}
\end{itemize}
%\input{part1/vorarbeiten}

\chapter{Firmenstandards}
\section{Bootstrapping}


\section{IP-Adressen}


\chapter{Organisation der IPA}
\nopagebreak
\input{part1/organisation}

\chapter{Zeitplan}
%%%%%%%%%%%%%%%%%%%%%%%%%%%%%%%%%%%%%%%%%%%%%%%%%%%%%%%%%%%%%%%%%%%%%%%%%
%%                                                                  %%
%%  This is the header of a LaTeX2e file exported from Gnumeric.    %%
%%                                                                  %%
%%  This file can be compiled as it stands or included in another   %%
%%  LaTeX document. The table is based on the longtable package so  %%
%%  the longtable options (headers, footers...) can be set in the   %%
%%  preamble section below (see PRAMBLE).                           %%
%%                                                                  %%
%%  To include the file in another, the following two lines must be %%
%%  in the including file:                                          %%
%%        \def\inputGnumericTable{}                                 %%
%%  at the beginning of the file and:                               %%
%%        \input{name-of-this-file.tex}                             %%
%%  where the table is to be placed. Note also that the including   %%
%%  file must use the following packages for the table to be        %%
%%  rendered correctly:                                             %%
%%    \usepackage[latin1]{inputenc}                                 %%
%%    \usepackage{color}                                            %%
%%    \usepackage{array}                                            %%
%%    \usepackage{longtable}                                        %%
%%    \usepackage{calc}                                             %%
%%    \usepackage{multirow}                                         %%
%%    \usepackage{hhline}                                           %%
%%    \usepackage{ifthen}                                           %%
%%  optionally (for landscape tables embedded in another document): %%
%%    \usepackage{lscape}                                           %%
%%                                                                  %%
%%%%%%%%%%%%%%%%%%%%%%%%%%%%%%%%%%%%%%%%%%%%%%%%%%%%%%%%%%%%%%%%%%%%%%



%%  This section checks if we are begin input into another file or  %%
%%  the file will be compiled alone. First use a macro taken from   %%
%%  the TeXbook ex 7.7 (suggestion of Han-Wen Nienhuys).            %%
\def\ifundefined#1{\expandafter\ifx\csname#1\endcsname\relax}


%%  Check for the \def token for inputed files. If it is not        %%
%%  defined, the file will be processed as a standalone and the     %%
%%  preamble will be used.                                          %%
\ifundefined{inputGnumericTable}

%%  We must be able to close or not the document at the end.        %%
	\def\gnumericTableEnd{\end{document}}


%%%%%%%%%%%%%%%%%%%%%%%%%%%%%%%%%%%%%%%%%%%%%%%%%%%%%%%%%%%%%%%%%%%%%%
%%                                                                  %%
%%  This is the PREAMBLE. Change these values to get the right      %%
%%  paper size and other niceties. Uncomment the landscape option   %%
%%  to the documentclass defintion for standalone documents.        %%
%%                                                                  %%
%%%%%%%%%%%%%%%%%%%%%%%%%%%%%%%%%%%%%%%%%%%%%%%%%%%%%%%%%%%%%%%%%%%%%%

	\documentclass[12pt%
			  %,landscape%
                    ]{report}
       \usepackage[latin1]{inputenc}
	\usepackage{fullpage}
	\usepackage{color}
       \usepackage{array}
	\usepackage{longtable}
       \usepackage{calc}
       \usepackage{multirow}
       \usepackage{hhline}
       \usepackage{ifthen}

	\begin{document}


%%  End of the preamble for the standalone. The next section is for %%
%%  documents which are included into other LaTeX2e files.          %%
\else

%%  We are not a stand alone document. For a regular table, we will %%
%%  have no preamble and only define the closing to mean nothing.   %%
    \def\gnumericTableEnd{}

%%  If we want landscape mode in an embedded document, comment out  %%
%%  the line above and uncomment the two below. The table will      %%
%%  begin on a new page and run in landscape mode.                  %%
%       \def\gnumericTableEnd{\end{landscape}}
%       \begin{landscape}


%%  End of the else clause for this file being \input.              %%
\fi

%%%%%%%%%%%%%%%%%%%%%%%%%%%%%%%%%%%%%%%%%%%%%%%%%%%%%%%%%%%%%%%%%%%%%%
%%                                                                  %%
%%  The rest is the gnumeric table, except for the closing          %%
%%  statement. Changes below will alter the table's appearance.     %%
%%                                                                  %%
%%%%%%%%%%%%%%%%%%%%%%%%%%%%%%%%%%%%%%%%%%%%%%%%%%%%%%%%%%%%%%%%%%%%%%

\providecommand{\gnumericmathit}[1]{#1} 
%%  Uncomment the next line if you would like your numbers to be in %%
%%  italics if they are italizised in the gnumeric table.           %%
%\renewcommand{\gnumericmathit}[1]{\mathit{#1}}
\providecommand{\gnumericPB}[1]%
{\let\gnumericTemp=\\#1\let\\=\gnumericTemp\hspace{0pt}}
 \ifundefined{gnumericTableWidthDefined}
        \newlength{\gnumericTableWidth}
        \newlength{\gnumericTableWidthComplete}
        \newlength{\gnumericMultiRowLength}
        \global\def\gnumericTableWidthDefined{}
 \fi
%% The following setting protects this code from babel shorthands.  %%
 \ifthenelse{\isundefined{\languageshorthands}}{}{\languageshorthands{english}}
%%  The default table format retains the relative column widths of  %%
%%  gnumeric. They can easily be changed to c, r or l. In that case %%
%%  you may want to comment out the next line and uncomment the one %%
%%  thereafter                                                      %%
\providecommand\gnumbox{\makebox[0pt]}
%%\providecommand\gnumbox[1][]{\makebox}

%% to adjust positions in multirow situations                       %%
\setlength{\bigstrutjot}{\jot}
\setlength{\extrarowheight}{\doublerulesep}

%%  The \setlongtables command keeps column widths the same across  %%
%%  pages. Simply comment out next line for varying column widths.  %%
\setlongtables

\setlength\gnumericTableWidth{%
	170pt+%
	83pt+%
	39pt+%
	39pt+%
	39pt+%
	39pt+%
	39pt+%
	39pt+%
	39pt+%
	39pt+%
	39pt+%
	39pt+%
	39pt+%
	39pt+%
	39pt+%
	39pt+%
	39pt+%
	39pt+%
	39pt+%
	39pt+%
	39pt+%
	39pt+%
	32pt+%
	50pt+%
	43pt+%
0pt}
\def\gumericNumCols{25}
\setlength\gnumericTableWidthComplete{\gnumericTableWidth+%
         \tabcolsep*\gumericNumCols*2+\arrayrulewidth*\gumericNumCols}
\ifthenelse{\lengthtest{\gnumericTableWidthComplete > \linewidth}}%
         {\def\gnumericScale{\ratio{\linewidth-%
                        \tabcolsep*\gumericNumCols*2-%
                        \arrayrulewidth*\gumericNumCols}%
{\gnumericTableWidth}}}%
{\def\gnumericScale{1}}

%%%%%%%%%%%%%%%%%%%%%%%%%%%%%%%%%%%%%%%%%%%%%%%%%%%%%%%%%%%%%%%%%%%%%%
%%                                                                  %%
%% The following are the widths of the various columns. We are      %%
%% defining them here because then they are easier to change.       %%
%% Depending on the cell formats we may use them more than once.    %%
%%                                                                  %%
%%%%%%%%%%%%%%%%%%%%%%%%%%%%%%%%%%%%%%%%%%%%%%%%%%%%%%%%%%%%%%%%%%%%%%

\ifthenelse{\isundefined{\gnumericColA}}{\newlength{\gnumericColA}}{}\settowidth{\gnumericColA}{\begin{tabular}{@{}p{170pt*\gnumericScale}@{}}x\end{tabular}}
\ifthenelse{\isundefined{\gnumericColB}}{\newlength{\gnumericColB}}{}\settowidth{\gnumericColB}{\begin{tabular}{@{}p{83pt*\gnumericScale}@{}}x\end{tabular}}
\ifthenelse{\isundefined{\gnumericColC}}{\newlength{\gnumericColC}}{}\settowidth{\gnumericColC}{\begin{tabular}{@{}p{39pt*\gnumericScale}@{}}x\end{tabular}}
\ifthenelse{\isundefined{\gnumericColD}}{\newlength{\gnumericColD}}{}\settowidth{\gnumericColD}{\begin{tabular}{@{}p{39pt*\gnumericScale}@{}}x\end{tabular}}
\ifthenelse{\isundefined{\gnumericColE}}{\newlength{\gnumericColE}}{}\settowidth{\gnumericColE}{\begin{tabular}{@{}p{39pt*\gnumericScale}@{}}x\end{tabular}}
\ifthenelse{\isundefined{\gnumericColF}}{\newlength{\gnumericColF}}{}\settowidth{\gnumericColF}{\begin{tabular}{@{}p{39pt*\gnumericScale}@{}}x\end{tabular}}
\ifthenelse{\isundefined{\gnumericColG}}{\newlength{\gnumericColG}}{}\settowidth{\gnumericColG}{\begin{tabular}{@{}p{39pt*\gnumericScale}@{}}x\end{tabular}}
\ifthenelse{\isundefined{\gnumericColH}}{\newlength{\gnumericColH}}{}\settowidth{\gnumericColH}{\begin{tabular}{@{}p{39pt*\gnumericScale}@{}}x\end{tabular}}
\ifthenelse{\isundefined{\gnumericColI}}{\newlength{\gnumericColI}}{}\settowidth{\gnumericColI}{\begin{tabular}{@{}p{39pt*\gnumericScale}@{}}x\end{tabular}}
\ifthenelse{\isundefined{\gnumericColJ}}{\newlength{\gnumericColJ}}{}\settowidth{\gnumericColJ}{\begin{tabular}{@{}p{39pt*\gnumericScale}@{}}x\end{tabular}}
\ifthenelse{\isundefined{\gnumericColK}}{\newlength{\gnumericColK}}{}\settowidth{\gnumericColK}{\begin{tabular}{@{}p{39pt*\gnumericScale}@{}}x\end{tabular}}
\ifthenelse{\isundefined{\gnumericColL}}{\newlength{\gnumericColL}}{}\settowidth{\gnumericColL}{\begin{tabular}{@{}p{39pt*\gnumericScale}@{}}x\end{tabular}}
\ifthenelse{\isundefined{\gnumericColM}}{\newlength{\gnumericColM}}{}\settowidth{\gnumericColM}{\begin{tabular}{@{}p{39pt*\gnumericScale}@{}}x\end{tabular}}
\ifthenelse{\isundefined{\gnumericColN}}{\newlength{\gnumericColN}}{}\settowidth{\gnumericColN}{\begin{tabular}{@{}p{39pt*\gnumericScale}@{}}x\end{tabular}}
\ifthenelse{\isundefined{\gnumericColO}}{\newlength{\gnumericColO}}{}\settowidth{\gnumericColO}{\begin{tabular}{@{}p{39pt*\gnumericScale}@{}}x\end{tabular}}
\ifthenelse{\isundefined{\gnumericColP}}{\newlength{\gnumericColP}}{}\settowidth{\gnumericColP}{\begin{tabular}{@{}p{39pt*\gnumericScale}@{}}x\end{tabular}}
\ifthenelse{\isundefined{\gnumericColQ}}{\newlength{\gnumericColQ}}{}\settowidth{\gnumericColQ}{\begin{tabular}{@{}p{39pt*\gnumericScale}@{}}x\end{tabular}}
\ifthenelse{\isundefined{\gnumericColR}}{\newlength{\gnumericColR}}{}\settowidth{\gnumericColR}{\begin{tabular}{@{}p{39pt*\gnumericScale}@{}}x\end{tabular}}
\ifthenelse{\isundefined{\gnumericColS}}{\newlength{\gnumericColS}}{}\settowidth{\gnumericColS}{\begin{tabular}{@{}p{39pt*\gnumericScale}@{}}x\end{tabular}}
\ifthenelse{\isundefined{\gnumericColT}}{\newlength{\gnumericColT}}{}\settowidth{\gnumericColT}{\begin{tabular}{@{}p{39pt*\gnumericScale}@{}}x\end{tabular}}
\ifthenelse{\isundefined{\gnumericColU}}{\newlength{\gnumericColU}}{}\settowidth{\gnumericColU}{\begin{tabular}{@{}p{39pt*\gnumericScale}@{}}x\end{tabular}}
\ifthenelse{\isundefined{\gnumericColV}}{\newlength{\gnumericColV}}{}\settowidth{\gnumericColV}{\begin{tabular}{@{}p{39pt*\gnumericScale}@{}}x\end{tabular}}
\ifthenelse{\isundefined{\gnumericColW}}{\newlength{\gnumericColW}}{}\settowidth{\gnumericColW}{\begin{tabular}{@{}p{32pt*\gnumericScale}@{}}x\end{tabular}}
\ifthenelse{\isundefined{\gnumericColX}}{\newlength{\gnumericColX}}{}\settowidth{\gnumericColX}{\begin{tabular}{@{}p{50pt*\gnumericScale}@{}}x\end{tabular}}
\ifthenelse{\isundefined{\gnumericColY}}{\newlength{\gnumericColY}}{}\settowidth{\gnumericColY}{\begin{tabular}{@{}p{43pt*\gnumericScale}@{}}x\end{tabular}}

\begin{longtable}[c]{%
	b{\gnumericColA}%
	b{\gnumericColB}%
	b{\gnumericColC}%
	b{\gnumericColD}%
	b{\gnumericColE}%
	b{\gnumericColF}%
	b{\gnumericColG}%
	b{\gnumericColH}%
	b{\gnumericColI}%
	b{\gnumericColJ}%
	b{\gnumericColK}%
	b{\gnumericColL}%
	b{\gnumericColM}%
	b{\gnumericColN}%
	b{\gnumericColO}%
	b{\gnumericColP}%
	b{\gnumericColQ}%
	b{\gnumericColR}%
	b{\gnumericColS}%
	b{\gnumericColT}%
	b{\gnumericColU}%
	b{\gnumericColV}%
	b{\gnumericColW}%
	b{\gnumericColX}%
	b{\gnumericColY}%
	}

%%%%%%%%%%%%%%%%%%%%%%%%%%%%%%%%%%%%%%%%%%%%%%%%%%%%%%%%%%%%%%%%%%%%%%
%%  The longtable options. (Caption, headers... see Goosens, p.124) %%
%	\caption{The Table Caption.}             \\	%
% \hline	% Across the top of the table.
%%  The rest of these options are table rows which are placed on    %%
%%  the first, last or every page. Use \multicolumn if you want.    %%

%%  Header for the first page.                                      %%
%	\multicolumn{25}{c}{The First Header} \\ \hline 
%	\multicolumn{1}{c}{colTag}	%Column 1
%	&\multicolumn{1}{c}{colTag}	%Column 2
%	&\multicolumn{1}{c}{colTag}	%Column 3
%	&\multicolumn{1}{c}{colTag}	%Column 4
%	&\multicolumn{1}{c}{colTag}	%Column 5
%	&\multicolumn{1}{c}{colTag}	%Column 6
%	&\multicolumn{1}{c}{colTag}	%Column 7
%	&\multicolumn{1}{c}{colTag}	%Column 8
%	&\multicolumn{1}{c}{colTag}	%Column 9
%	&\multicolumn{1}{c}{colTag}	%Column 10
%	&\multicolumn{1}{c}{colTag}	%Column 11
%	&\multicolumn{1}{c}{colTag}	%Column 12
%	&\multicolumn{1}{c}{colTag}	%Column 13
%	&\multicolumn{1}{c}{colTag}	%Column 14
%	&\multicolumn{1}{c}{colTag}	%Column 15
%	&\multicolumn{1}{c}{colTag}	%Column 16
%	&\multicolumn{1}{c}{colTag}	%Column 17
%	&\multicolumn{1}{c}{colTag}	%Column 18
%	&\multicolumn{1}{c}{colTag}	%Column 19
%	&\multicolumn{1}{c}{colTag}	%Column 20
%	&\multicolumn{1}{c}{colTag}	%Column 21
%	&\multicolumn{1}{c}{colTag}	%Column 22
%	&\multicolumn{1}{c}{colTag}	%Column 23
%	&\multicolumn{1}{c}{colTag}	%Column 24
%	&\multicolumn{1}{c}{colTag}	\\ \hline %Last column
%	\endfirsthead

%%  The running header definition.                                  %%
%	\hline
%	\multicolumn{25}{l}{\ldots\small\slshape continued} \\ \hline
%	\multicolumn{1}{c}{colTag}	%Column 1
%	&\multicolumn{1}{c}{colTag}	%Column 2
%	&\multicolumn{1}{c}{colTag}	%Column 3
%	&\multicolumn{1}{c}{colTag}	%Column 4
%	&\multicolumn{1}{c}{colTag}	%Column 5
%	&\multicolumn{1}{c}{colTag}	%Column 6
%	&\multicolumn{1}{c}{colTag}	%Column 7
%	&\multicolumn{1}{c}{colTag}	%Column 8
%	&\multicolumn{1}{c}{colTag}	%Column 9
%	&\multicolumn{1}{c}{colTag}	%Column 10
%	&\multicolumn{1}{c}{colTag}	%Column 11
%	&\multicolumn{1}{c}{colTag}	%Column 12
%	&\multicolumn{1}{c}{colTag}	%Column 13
%	&\multicolumn{1}{c}{colTag}	%Column 14
%	&\multicolumn{1}{c}{colTag}	%Column 15
%	&\multicolumn{1}{c}{colTag}	%Column 16
%	&\multicolumn{1}{c}{colTag}	%Column 17
%	&\multicolumn{1}{c}{colTag}	%Column 18
%	&\multicolumn{1}{c}{colTag}	%Column 19
%	&\multicolumn{1}{c}{colTag}	%Column 20
%	&\multicolumn{1}{c}{colTag}	%Column 21
%	&\multicolumn{1}{c}{colTag}	%Column 22
%	&\multicolumn{1}{c}{colTag}	%Column 23
%	&\multicolumn{1}{c}{colTag}	%Column 24
%	&\multicolumn{1}{c}{colTag}	\\ \hline %Last column
%	\endhead

%%  The running footer definition.                                  %%
%	\hline
%	\multicolumn{25}{r}{\small\slshape continued\ldots} \\
%	\endfoot

%%  The ending footer definition.                                   %%
%	\multicolumn{25}{c}{That's all folks} \\ \hline 
%	\endlastfoot
%%%%%%%%%%%%%%%%%%%%%%%%%%%%%%%%%%%%%%%%%%%%%%%%%%%%%%%%%%%%%%%%%%%%%%

\hhline{~-|-|-|-|-|-|-|-|-|-|-|-|-|-|-|-|-|-|-|-|-~|-~}
	 
	&\multicolumn{1}{p{\gnumericColB}|}%
	{}
	&\multicolumn{2}{p{	\gnumericColC+%
	\gnumericColD+%
	\tabcolsep*2*1}|}%
	{\gnumericPB{\centering}\gnumbox{{\color[rgb]{0.00,0.50,0.00} Tag 1
Di, 11.02.}}}
	&\multicolumn{2}{p{	\gnumericColE+%
	\gnumericColF+%
	\tabcolsep*2*1}|}%
	{\gnumericPB{\centering}\gnumbox{{\color[rgb]{0.00,0.50,0.00} Tag 2
Mi, 12.02.}}}
	&\multicolumn{2}{p{	\gnumericColG+%
	\gnumericColH+%
	\tabcolsep*2*1}|}%
	{\gnumericPB{\centering}\gnumbox{{\color[rgb]{0.00,0.50,0.00} Tag 3
Do, 13.02.}}}
	&\multicolumn{2}{p{	\gnumericColI+%
	\gnumericColJ+%
	\tabcolsep*2*1}|}%
	{\gnumericPB{\centering}\gnumbox{{\color[rgb]{0.00,0.50,0.00} Tag 4
Fr, 14.02.}}}
	&\multicolumn{2}{p{	\gnumericColK+%
	\gnumericColL+%
	\tabcolsep*2*1}|}%
	{\gnumericPB{\centering}\gnumbox{{\color[rgb]{0.00,0.50,0.00} Tag 5
Di, 18.02.}}}
	&\multicolumn{2}{p{	\gnumericColM+%
	\gnumericColN+%
	\tabcolsep*2*1}|}%
	{\gnumericPB{\centering}\gnumbox{{\color[rgb]{0.00,0.50,0.00} Tag 6
Mi, 19.02}}}
	&\multicolumn{2}{p{	\gnumericColO+%
	\gnumericColP+%
	\tabcolsep*2*1}|}%
	{\gnumericPB{\centering}\gnumbox{{\color[rgb]{0.00,0.50,0.00} Tag 7
Do, 20.02}}}
	&\multicolumn{2}{p{	\gnumericColQ+%
	\gnumericColR+%
	\tabcolsep*2*1}|}%
	{\gnumericPB{\centering}\gnumbox{{\color[rgb]{0.00,0.50,0.00} Tag 8
Fr, 21.02.}}}
	&\multicolumn{2}{p{	\gnumericColS+%
	\gnumericColT+%
	\tabcolsep*2*1}|}%
	{\gnumericPB{\centering}\gnumbox{{\color[rgb]{0.00,0.50,0.00} Tag 9
Di, 25.02}}}
	&\multicolumn{2}{p{	\gnumericColU+%
	\gnumericColV+%
	\tabcolsep*2*1}|}%
	{\gnumericPB{\centering}\gnumbox{{\color[rgb]{0.00,0.50,0.00} Tag 10
Mi, 26.02.}}}
	&\multicolumn{1}{p{\gnumericColW}|}%
	{}
	&\multicolumn{1}{p{\gnumericColX}|}%
	{}
	&
\\
\hhline{~---------------------|~|-|~}
	 
	&\multicolumn{1}{p{\gnumericColB}|}%
	{}
	&\multicolumn{1}{p{\gnumericColC}|}%
	{\gnumericPB{\centering}\gnumbox{{\color[rgb]{0.00,0.50,0.00} V}}}
	&\multicolumn{1}{p{\gnumericColD}|}%
	{\gnumericPB{\centering}\gnumbox{{\color[rgb]{0.00,0.50,0.00} N}}}
	&\multicolumn{1}{p{\gnumericColE}|}%
	{\gnumericPB{\centering}\gnumbox{{\color[rgb]{0.00,0.50,0.00} V}}}
	&\multicolumn{1}{p{\gnumericColF}|}%
	{\gnumericPB{\centering}\gnumbox{{\color[rgb]{0.00,0.50,0.00} N}}}
	&\multicolumn{1}{p{\gnumericColG}|}%
	{\gnumericPB{\centering}\gnumbox{{\color[rgb]{0.00,0.50,0.00} V}}}
	&\multicolumn{1}{p{\gnumericColH}|}%
	{\gnumericPB{\centering}\gnumbox{{\color[rgb]{0.00,0.50,0.00} N}}}
	&\multicolumn{1}{p{\gnumericColI}|}%
	{\gnumericPB{\centering}\gnumbox{{\color[rgb]{0.00,0.50,0.00} V}}}
	&\multicolumn{1}{p{\gnumericColJ}|}%
	{\gnumericPB{\centering}\gnumbox{{\color[rgb]{0.00,0.50,0.00} N}}}
	&\multicolumn{1}{p{\gnumericColK}|}%
	{\gnumericPB{\centering}\gnumbox{{\color[rgb]{0.00,0.50,0.00} V}}}
	&\multicolumn{1}{p{\gnumericColL}|}%
	{\gnumericPB{\centering}\gnumbox{{\color[rgb]{0.00,0.50,0.00} N}}}
	&\multicolumn{1}{p{\gnumericColM}|}%
	{\gnumericPB{\centering}\gnumbox{{\color[rgb]{0.00,0.50,0.00} V}}}
	&\multicolumn{1}{p{\gnumericColN}|}%
	{\gnumericPB{\centering}\gnumbox{{\color[rgb]{0.00,0.50,0.00} N}}}
	&\multicolumn{1}{p{\gnumericColO}|}%
	{\gnumericPB{\centering}\gnumbox{{\color[rgb]{0.00,0.50,0.00} V}}}
	&\multicolumn{1}{p{\gnumericColP}|}%
	{\gnumericPB{\centering}\gnumbox{{\color[rgb]{0.00,0.50,0.00} N}}}
	&\multicolumn{1}{p{\gnumericColQ}|}%
	{\gnumericPB{\centering}\gnumbox{{\color[rgb]{0.00,0.50,0.00} V}}}
	&\multicolumn{1}{p{\gnumericColR}|}%
	{\gnumericPB{\centering}\gnumbox{{\color[rgb]{0.00,0.50,0.00} N}}}
	&\multicolumn{1}{p{\gnumericColS}|}%
	{\gnumericPB{\centering}\gnumbox{{\color[rgb]{0.00,0.50,0.00} V}}}
	&\multicolumn{1}{p{\gnumericColT}|}%
	{\gnumericPB{\centering}\gnumbox{{\color[rgb]{0.00,0.50,0.00} N}}}
	&\multicolumn{1}{p{\gnumericColU}|}%
	{\gnumericPB{\centering}\gnumbox{{\color[rgb]{0.00,0.50,0.00} V}}}
	&\multicolumn{1}{p{\gnumericColV}|}%
	{\gnumericPB{\centering}\gnumbox{{\color[rgb]{0.00,0.50,0.00} N}}}
	&\multicolumn{1}{p{\gnumericColW}|}%
	{}
	&\multicolumn{1}{p{\gnumericColX}|}%
	{\setlength{\gnumericMultiRowLength}{0pt}%
	 \addtolength{\gnumericMultiRowLength}{\gnumericColX}%
	 \multirow{3}[1]{\gnumericMultiRowLength}{\parbox{\gnumericMultiRowLength}{%
	 \gnumericPB{\centering}\gnumbox{{\color[rgb]{0.00,0.50,0.00} Total Soll}}}}}
	&\setlength{\gnumericMultiRowLength}{0pt}%
	 \addtolength{\gnumericMultiRowLength}{\gnumericColY}%
	 \multirow{3}[1]{\gnumericMultiRowLength}{\parbox{\gnumericMultiRowLength}{%
	 \gnumericPB{\centering}\gnumbox{Total Ist}}}
\\
\hhline{~---------------------|~|-|~}
	 
	&\multicolumn{1}{p{\gnumericColB}|}%
	{\gnumericPB{\centering}\gnumbox{{\color[rgb]{0.00,0.50,0.00} \textbf{\textit{Phasen}}}}}
	&\multicolumn{3}{p{	\gnumericColC+%
	\gnumericColD+%
	\gnumericColE+%
	\tabcolsep*2*2}|}%
	{\gnumericPB{\centering}\gnumbox{{\color[rgb]{0.00,0.50,0.00} \textit{Start / Grundsystem}}}}
	&\multicolumn{2}{p{	\gnumericColF+%
	\gnumericColG+%
	\tabcolsep*2*1}|}%
	{\gnumericPB{\centering}\gnumbox{{\color[rgb]{0.00,0.50,0.00} \textit{DRBD}}}}
	&\multicolumn{3}{p{	\gnumericColH+%
	\gnumericColI+%
	\gnumericColJ+%
	\tabcolsep*2*2}|}%
	{\gnumericPB{\centering}\gnumbox{{\color[rgb]{0.00,0.50,0.00} \textit{Pacemaker}}}}
	&\multicolumn{4}{p{	\gnumericColK+%
	\gnumericColL+%
	\gnumericColM+%
	\gnumericColN+%
	\tabcolsep*2*3}|}%
	{\gnumericPB{\centering}\gnumbox{{\color[rgb]{0.00,0.50,0.00} \textit{RHEV}}}}
	&\multicolumn{2}{p{	\gnumericColO+%
	\gnumericColP+%
	\tabcolsep*2*1}|}%
	{\gnumericPB{\centering}\gnumbox{{\color[rgb]{0.00,0.50,0.00} \textit{Doku}}}}
	&\multicolumn{4}{p{	\gnumericColQ+%
	\gnumericColR+%
	\gnumericColS+%
	\gnumericColT+%
	\tabcolsep*2*3}|}%
	{\gnumericPB{\centering}\gnumbox{{\color[rgb]{0.00,0.50,0.00} \textit{Testing}}}}
	&\multicolumn{2}{p{	\gnumericColU+%
	\gnumericColV+%
	\tabcolsep*2*1}|}%
	{\gnumericPB{\centering}\gnumbox{{\color[rgb]{0.00,0.50,0.00} \textit{Abschluss}}}}
	&\multicolumn{1}{p{\gnumericColW}|}%
	{}
	&\multicolumn{1}{p{\gnumericColX}|}%
	{}
	&
\\
\hhline{~---------------------|~|-|~}
	 \gnumericPB{\raggedright}\gnumbox[l]{\textbf{Aufgaben}}
	&\multicolumn{1}{p{\gnumericColB}|}%
	{}
	&\multicolumn{1}{p{\gnumericColC}|}%
	{}
	&\multicolumn{1}{p{\gnumericColD}|}%
	{}
	&\multicolumn{1}{p{\gnumericColE}|}%
	{}
	&\multicolumn{1}{p{\gnumericColF}|}%
	{}
	&\multicolumn{1}{p{\gnumericColG}|}%
	{}
	&\multicolumn{1}{p{\gnumericColH}|}%
	{}
	&\multicolumn{1}{p{\gnumericColI}|}%
	{}
	&\multicolumn{1}{p{\gnumericColJ}|}%
	{}
	&\multicolumn{1}{p{\gnumericColK}|}%
	{}
	&\multicolumn{1}{p{\gnumericColL}|}%
	{}
	&\multicolumn{1}{p{\gnumericColM}|}%
	{}
	&\multicolumn{1}{p{\gnumericColN}|}%
	{}
	&\multicolumn{1}{p{\gnumericColO}|}%
	{}
	&\multicolumn{1}{p{\gnumericColP}|}%
	{}
	&\multicolumn{1}{p{\gnumericColQ}|}%
	{}
	&\multicolumn{1}{p{\gnumericColR}|}%
	{}
	&\multicolumn{1}{p{\gnumericColS}|}%
	{}
	&\multicolumn{1}{p{\gnumericColT}|}%
	{}
	&\multicolumn{1}{p{\gnumericColU}|}%
	{}
	&\multicolumn{1}{p{\gnumericColV}|}%
	{}
	&\multicolumn{1}{p{\gnumericColW}|}%
	{}
	&\multicolumn{1}{p{\gnumericColX}|}%
	{}
	&
\\
\hhline{~---------------------|~|-|~}
	 \gnumericPB{\raggedright}\gnumbox[l]{\textit{Projektmanagement}}
	&\multicolumn{1}{p{\gnumericColB}|}%
	{}
	&\multicolumn{1}{p{\gnumericColC}|}%
	{}
	&\multicolumn{1}{p{\gnumericColD}|}%
	{}
	&\multicolumn{1}{p{\gnumericColE}|}%
	{}
	&\multicolumn{1}{p{\gnumericColF}|}%
	{}
	&\multicolumn{1}{p{\gnumericColG}|}%
	{}
	&\multicolumn{1}{p{\gnumericColH}|}%
	{}
	&\multicolumn{1}{p{\gnumericColI}|}%
	{}
	&\multicolumn{1}{p{\gnumericColJ}|}%
	{}
	&\multicolumn{1}{p{\gnumericColK}|}%
	{}
	&\multicolumn{1}{p{\gnumericColL}|}%
	{}
	&\multicolumn{1}{p{\gnumericColM}|}%
	{}
	&\multicolumn{1}{p{\gnumericColN}|}%
	{}
	&\multicolumn{1}{p{\gnumericColO}|}%
	{}
	&\multicolumn{1}{p{\gnumericColP}|}%
	{}
	&\multicolumn{1}{p{\gnumericColQ}|}%
	{}
	&\multicolumn{1}{p{\gnumericColR}|}%
	{}
	&\multicolumn{1}{p{\gnumericColS}|}%
	{}
	&\multicolumn{1}{p{\gnumericColT}|}%
	{}
	&\multicolumn{1}{p{\gnumericColU}|}%
	{}
	&\multicolumn{1}{p{\gnumericColV}|}%
	{}
	&\multicolumn{1}{p{\gnumericColW}|}%
	{}
	&\multicolumn{1}{p{\gnumericColX}|}%
	{\gnumericPB{\centering}\gnumbox{{\color[rgb]{0.00,0.50,0.00} \textit{46}}}}
	&\gnumericPB{\centering}\gnumbox{\textit{2}}
\\
\hhline{~---------------------|~|-|~}
	 \gnumericPB{\raggedright}\gnumbox[l]{Tagesplan / Daily Scrum}
	&\multicolumn{1}{p{\gnumericColB}|}%
	{\gnumericPB{\centering}\gnumbox{{\color[rgb]{0.00,0.50,0.00} Soll}}}
	&\multicolumn{1}{p{\gnumericColC}|}%
	{\gnumericPB{\raggedleft}\gnumbox[r]{{\color[rgb]{0.00,0.50,0.00} 1}}}
	&\multicolumn{1}{p{\gnumericColD}|}%
	{}
	&\multicolumn{1}{p{\gnumericColE}|}%
	{\gnumericPB{\raggedleft}\gnumbox[r]{{\color[rgb]{0.00,0.50,0.00} 1}}}
	&\multicolumn{1}{p{\gnumericColF}|}%
	{}
	&\multicolumn{1}{p{\gnumericColG}|}%
	{\gnumericPB{\raggedleft}\gnumbox[r]{{\color[rgb]{0.00,0.50,0.00} 1}}}
	&\multicolumn{1}{p{\gnumericColH}|}%
	{}
	&\multicolumn{1}{p{\gnumericColI}|}%
	{\gnumericPB{\raggedleft}\gnumbox[r]{{\color[rgb]{0.00,0.50,0.00} 1}}}
	&\multicolumn{1}{p{\gnumericColJ}|}%
	{}
	&\multicolumn{1}{p{\gnumericColK}|}%
	{\gnumericPB{\raggedleft}\gnumbox[r]{{\color[rgb]{0.00,0.50,0.00} 1}}}
	&\multicolumn{1}{p{\gnumericColL}|}%
	{}
	&\multicolumn{1}{p{\gnumericColM}|}%
	{\gnumericPB{\raggedleft}\gnumbox[r]{{\color[rgb]{0.00,0.50,0.00} 1}}}
	&\multicolumn{1}{p{\gnumericColN}|}%
	{}
	&\multicolumn{1}{p{\gnumericColO}|}%
	{\gnumericPB{\raggedleft}\gnumbox[r]{{\color[rgb]{0.00,0.50,0.00} 1}}}
	&\multicolumn{1}{p{\gnumericColP}|}%
	{}
	&\multicolumn{1}{p{\gnumericColQ}|}%
	{\gnumericPB{\raggedleft}\gnumbox[r]{{\color[rgb]{0.00,0.50,0.00} 1}}}
	&\multicolumn{1}{p{\gnumericColR}|}%
	{}
	&\multicolumn{1}{p{\gnumericColS}|}%
	{\gnumericPB{\raggedleft}\gnumbox[r]{{\color[rgb]{0.00,0.50,0.00} 1}}}
	&\multicolumn{1}{p{\gnumericColT}|}%
	{}
	&\multicolumn{1}{p{\gnumericColU}|}%
	{\gnumericPB{\raggedleft}\gnumbox[r]{{\color[rgb]{0.00,0.50,0.00} 1}}}
	&\multicolumn{1}{p{\gnumericColV}|}%
	{}
	&\multicolumn{1}{p{\gnumericColW}|}%
	{}
	&\multicolumn{1}{p{\gnumericColX}|}%
	{\gnumericPB{\raggedleft}\gnumbox[r]{{\color[rgb]{0.00,0.50,0.00} 10}}}
	&
\\
\hhline{~---------------------|~|--}
	 
	&\multicolumn{1}{p{\gnumericColB}|}%
	{\gnumericPB{\centering}\gnumbox{{\color[rgb]{0.00,0.50,0.00} \textbf{\textit{Ist}}}}}
	&\multicolumn{1}{p{\gnumericColC}|}%
	{\gnumericPB{\raggedleft}\gnumbox[r]{{\color[rgb]{0.00,0.50,0.00} \textbf{\textit{1}}}}}
	&\multicolumn{1}{p{\gnumericColD}|}%
	{}
	&\multicolumn{1}{p{\gnumericColE}|}%
	{}
	&\multicolumn{1}{p{\gnumericColF}|}%
	{}
	&\multicolumn{1}{p{\gnumericColG}|}%
	{}
	&\multicolumn{1}{p{\gnumericColH}|}%
	{}
	&\multicolumn{1}{p{\gnumericColI}|}%
	{}
	&\multicolumn{1}{p{\gnumericColJ}|}%
	{}
	&\multicolumn{1}{p{\gnumericColK}|}%
	{}
	&\multicolumn{1}{p{\gnumericColL}|}%
	{}
	&\multicolumn{1}{p{\gnumericColM}|}%
	{}
	&\multicolumn{1}{p{\gnumericColN}|}%
	{}
	&\multicolumn{1}{p{\gnumericColO}|}%
	{}
	&\multicolumn{1}{p{\gnumericColP}|}%
	{}
	&\multicolumn{1}{p{\gnumericColQ}|}%
	{}
	&\multicolumn{1}{p{\gnumericColR}|}%
	{}
	&\multicolumn{1}{p{\gnumericColS}|}%
	{}
	&\multicolumn{1}{p{\gnumericColT}|}%
	{}
	&\multicolumn{1}{p{\gnumericColU}|}%
	{}
	&\multicolumn{1}{p{\gnumericColV}|}%
	{}
	&\multicolumn{1}{p{\gnumericColW}|}%
	{}
	&\multicolumn{1}{p{\gnumericColX}|}%
	{}
	&\multicolumn{1}{p{\gnumericColY}|}%
	{\gnumericPB{\raggedleft}\gnumbox[r]{\textbf{\textit{1}}}}
\\
\hhline{~---------------------|~|--|}
	 \gnumericPB{\raggedright}\gnumbox[l]{Dokumentation Projekt}
	&\multicolumn{1}{p{\gnumericColB}|}%
	{\gnumericPB{\centering}\gnumbox{{\color[rgb]{0.00,0.50,0.00} Soll}}}
	&\multicolumn{1}{p{\gnumericColC}|}%
	{\gnumericPB{\raggedleft}\gnumbox[r]{{\color[rgb]{0.00,0.50,0.00} 1}}}
	&\multicolumn{1}{p{\gnumericColD}|}%
	{\gnumericPB{\raggedleft}\gnumbox[r]{{\color[rgb]{0.00,0.50,0.00} 4}}}
	&\multicolumn{1}{p{\gnumericColE}|}%
	{}
	&\multicolumn{1}{p{\gnumericColF}|}%
	{\gnumericPB{\raggedleft}\gnumbox[r]{{\color[rgb]{0.00,0.50,0.00} 1}}}
	&\multicolumn{1}{p{\gnumericColG}|}%
	{}
	&\multicolumn{1}{p{\gnumericColH}|}%
	{\gnumericPB{\raggedleft}\gnumbox[r]{{\color[rgb]{0.00,0.50,0.00} 2}}}
	&\multicolumn{1}{p{\gnumericColI}|}%
	{}
	&\multicolumn{1}{p{\gnumericColJ}|}%
	{\gnumericPB{\raggedleft}\gnumbox[r]{{\color[rgb]{0.00,0.50,0.00} 1}}}
	&\multicolumn{1}{p{\gnumericColK}|}%
	{\gnumericPB{\raggedleft}\gnumbox[r]{{\color[rgb]{0.00,0.50,0.00} 2}}}
	&\multicolumn{1}{p{\gnumericColL}|}%
	{}
	&\multicolumn{1}{p{\gnumericColM}|}%
	{\gnumericPB{\raggedleft}\gnumbox[r]{{\color[rgb]{0.00,0.50,0.00} 1}}}
	&\multicolumn{1}{p{\gnumericColN}|}%
	{\gnumericPB{\raggedleft}\gnumbox[r]{{\color[rgb]{0.00,0.50,0.00} 1}}}
	&\multicolumn{1}{p{\gnumericColO}|}%
	{\gnumericPB{\raggedleft}\gnumbox[r]{{\color[rgb]{0.00,0.50,0.00} 3}}}
	&\multicolumn{1}{p{\gnumericColP}|}%
	{\gnumericPB{\raggedleft}\gnumbox[r]{{\color[rgb]{0.00,0.50,0.00} 4}}}
	&\multicolumn{1}{p{\gnumericColQ}|}%
	{}
	&\multicolumn{1}{p{\gnumericColR}|}%
	{\gnumericPB{\raggedleft}\gnumbox[r]{{\color[rgb]{0.00,0.50,0.00} 1}}}
	&\multicolumn{1}{p{\gnumericColS}|}%
	{}
	&\multicolumn{1}{p{\gnumericColT}|}%
	{\gnumericPB{\raggedleft}\gnumbox[r]{{\color[rgb]{0.00,0.50,0.00} 4}}}
	&\multicolumn{1}{p{\gnumericColU}|}%
	{\gnumericPB{\raggedleft}\gnumbox[r]{{\color[rgb]{0.00,0.50,0.00} 3}}}
	&\multicolumn{1}{p{\gnumericColV}|}%
	{\gnumericPB{\raggedleft}\gnumbox[r]{{\color[rgb]{0.00,0.50,0.00} 4}}}
	&\multicolumn{1}{p{\gnumericColW}|}%
	{}
	&\multicolumn{1}{p{\gnumericColX}|}%
	{\gnumericPB{\raggedleft}\gnumbox[r]{{\color[rgb]{0.00,0.50,0.00} 32}}}
	&
\\
\hhline{~---------------------|~|--}
	 
	&\multicolumn{1}{p{\gnumericColB}|}%
	{\gnumericPB{\centering}\gnumbox{{\color[rgb]{0.00,0.50,0.00} \textbf{\textit{Ist}}}}}
	&\multicolumn{1}{p{\gnumericColC}|}%
	{\gnumericPB{\raggedleft}\gnumbox[r]{{\color[rgb]{0.00,0.50,0.00} \textbf{\textit{1}}}}}
	&\multicolumn{1}{p{\gnumericColD}|}%
	{}
	&\multicolumn{1}{p{\gnumericColE}|}%
	{}
	&\multicolumn{1}{p{\gnumericColF}|}%
	{}
	&\multicolumn{1}{p{\gnumericColG}|}%
	{}
	&\multicolumn{1}{p{\gnumericColH}|}%
	{}
	&\multicolumn{1}{p{\gnumericColI}|}%
	{}
	&\multicolumn{1}{p{\gnumericColJ}|}%
	{}
	&\multicolumn{1}{p{\gnumericColK}|}%
	{}
	&\multicolumn{1}{p{\gnumericColL}|}%
	{}
	&\multicolumn{1}{p{\gnumericColM}|}%
	{}
	&\multicolumn{1}{p{\gnumericColN}|}%
	{}
	&\multicolumn{1}{p{\gnumericColO}|}%
	{}
	&\multicolumn{1}{p{\gnumericColP}|}%
	{}
	&\multicolumn{1}{p{\gnumericColQ}|}%
	{}
	&\multicolumn{1}{p{\gnumericColR}|}%
	{}
	&\multicolumn{1}{p{\gnumericColS}|}%
	{}
	&\multicolumn{1}{p{\gnumericColT}|}%
	{}
	&\multicolumn{1}{p{\gnumericColU}|}%
	{}
	&\multicolumn{1}{p{\gnumericColV}|}%
	{}
	&\multicolumn{1}{p{\gnumericColW}|}%
	{}
	&\multicolumn{1}{p{\gnumericColX}|}%
	{}
	&\multicolumn{1}{p{\gnumericColY}|}%
	{\gnumericPB{\raggedleft}\gnumbox[r]{\textbf{\textit{1}}}}
\\
\hhline{~---------------------|~|--|}
	 \setlength{\gnumericMultiRowLength}{0pt}%
	 \addtolength{\gnumericMultiRowLength}{\gnumericColA}%
	 \multirow{2}[1]{\gnumericMultiRowLength}{\parbox{\gnumericMultiRowLength}{%
	 \gnumericPB{\raggedright}\gnumbox[l]{Review / Besprechung,
Expertenbesuch }}}
	&\multicolumn{1}{p{\gnumericColB}|}%
	{\gnumericPB{\centering}\gnumbox{{\color[rgb]{0.00,0.50,0.00} Soll}}}
	&\multicolumn{1}{p{\gnumericColC}|}%
	{}
	&\multicolumn{1}{p{\gnumericColD}|}%
	{}
	&\multicolumn{1}{p{\gnumericColE}|}%
	{}
	&\multicolumn{1}{p{\gnumericColF}|}%
	{\gnumericPB{\raggedleft}\gnumbox[r]{{\color[rgb]{0.00,0.50,0.00} 2}}}
	&\multicolumn{1}{p{\gnumericColG}|}%
	{}
	&\multicolumn{1}{p{\gnumericColH}|}%
	{}
	&\multicolumn{1}{p{\gnumericColI}|}%
	{}
	&\multicolumn{1}{p{\gnumericColJ}|}%
	{}
	&\multicolumn{1}{p{\gnumericColK}|}%
	{}
	&\multicolumn{1}{p{\gnumericColL}|}%
	{\gnumericPB{\raggedleft}\gnumbox[r]{{\color[rgb]{0.00,0.50,0.00} 2}}}
	&\multicolumn{1}{p{\gnumericColM}|}%
	{}
	&\multicolumn{1}{p{\gnumericColN}|}%
	{}
	&\multicolumn{1}{p{\gnumericColO}|}%
	{}
	&\multicolumn{1}{p{\gnumericColP}|}%
	{}
	&\multicolumn{1}{p{\gnumericColQ}|}%
	{}
	&\multicolumn{1}{p{\gnumericColR}|}%
	{}
	&\multicolumn{1}{p{\gnumericColS}|}%
	{}
	&\multicolumn{1}{p{\gnumericColT}|}%
	{}
	&\multicolumn{1}{p{\gnumericColU}|}%
	{}
	&\multicolumn{1}{p{\gnumericColV}|}%
	{}
	&\multicolumn{1}{p{\gnumericColW}|}%
	{}
	&\multicolumn{1}{p{\gnumericColX}|}%
	{\gnumericPB{\raggedleft}\gnumbox[r]{{\color[rgb]{0.00,0.50,0.00} 4}}}
	&
\\
\hhline{~---------------------|~|--}
	 
	&\multicolumn{1}{p{\gnumericColB}|}%
	{\gnumericPB{\centering}\gnumbox{{\color[rgb]{0.00,0.50,0.00} \textbf{\textit{Ist}}}}}
	&\multicolumn{1}{p{\gnumericColC}|}%
	{}
	&\multicolumn{1}{p{\gnumericColD}|}%
	{}
	&\multicolumn{1}{p{\gnumericColE}|}%
	{}
	&\multicolumn{1}{p{\gnumericColF}|}%
	{}
	&\multicolumn{1}{p{\gnumericColG}|}%
	{}
	&\multicolumn{1}{p{\gnumericColH}|}%
	{}
	&\multicolumn{1}{p{\gnumericColI}|}%
	{}
	&\multicolumn{1}{p{\gnumericColJ}|}%
	{}
	&\multicolumn{1}{p{\gnumericColK}|}%
	{}
	&\multicolumn{1}{p{\gnumericColL}|}%
	{}
	&\multicolumn{1}{p{\gnumericColM}|}%
	{}
	&\multicolumn{1}{p{\gnumericColN}|}%
	{}
	&\multicolumn{1}{p{\gnumericColO}|}%
	{}
	&\multicolumn{1}{p{\gnumericColP}|}%
	{}
	&\multicolumn{1}{p{\gnumericColQ}|}%
	{}
	&\multicolumn{1}{p{\gnumericColR}|}%
	{}
	&\multicolumn{1}{p{\gnumericColS}|}%
	{}
	&\multicolumn{1}{p{\gnumericColT}|}%
	{}
	&\multicolumn{1}{p{\gnumericColU}|}%
	{}
	&\multicolumn{1}{p{\gnumericColV}|}%
	{}
	&\multicolumn{1}{p{\gnumericColW}|}%
	{}
	&\multicolumn{1}{p{\gnumericColX}|}%
	{}
	&\multicolumn{1}{p{\gnumericColY}|}%
	{\gnumericPB{\raggedleft}\gnumbox[r]{\textbf{\textit{0}}}}
\\
\hhline{~---------------------|~|--|}
	 \gnumericPB{\raggedright}\gnumbox[l]{\textit{Umsetzung / Doku}}
	&\multicolumn{1}{p{\gnumericColB}|}%
	{}
	&\multicolumn{1}{p{\gnumericColC}|}%
	{}
	&\multicolumn{1}{p{\gnumericColD}|}%
	{}
	&\multicolumn{1}{p{\gnumericColE}|}%
	{}
	&\multicolumn{1}{p{\gnumericColF}|}%
	{}
	&\multicolumn{1}{p{\gnumericColG}|}%
	{}
	&\multicolumn{1}{p{\gnumericColH}|}%
	{}
	&\multicolumn{1}{p{\gnumericColI}|}%
	{}
	&\multicolumn{1}{p{\gnumericColJ}|}%
	{}
	&\multicolumn{1}{p{\gnumericColK}|}%
	{}
	&\multicolumn{1}{p{\gnumericColL}|}%
	{}
	&\multicolumn{1}{p{\gnumericColM}|}%
	{}
	&\multicolumn{1}{p{\gnumericColN}|}%
	{}
	&\multicolumn{1}{p{\gnumericColO}|}%
	{}
	&\multicolumn{1}{p{\gnumericColP}|}%
	{}
	&\multicolumn{1}{p{\gnumericColQ}|}%
	{}
	&\multicolumn{1}{p{\gnumericColR}|}%
	{}
	&\multicolumn{1}{p{\gnumericColS}|}%
	{}
	&\multicolumn{1}{p{\gnumericColT}|}%
	{}
	&\multicolumn{1}{p{\gnumericColU}|}%
	{}
	&\multicolumn{1}{p{\gnumericColV}|}%
	{}
	&\multicolumn{1}{p{\gnumericColW}|}%
	{}
	&\multicolumn{1}{p{\gnumericColX}|}%
	{\gnumericPB{\centering}\gnumbox{{\color[rgb]{0.00,0.50,0.00} \textit{34}}}}
	&\gnumericPB{\centering}\gnumbox{\textit{2}}
\\
\hhline{~---------------------|~|-|~}
	 \gnumericPB{\raggedright}\gnumbox[l]{Recherche, Vorbereitung}
	&\multicolumn{1}{p{\gnumericColB}|}%
	{\gnumericPB{\centering}\gnumbox{{\color[rgb]{0.00,0.50,0.00} Soll}}}
	&\multicolumn{1}{p{\gnumericColC}|}%
	{\gnumericPB{\raggedleft}\gnumbox[r]{{\color[rgb]{0.00,0.50,0.00} 2}}}
	&\multicolumn{1}{p{\gnumericColD}|}%
	{}
	&\multicolumn{1}{p{\gnumericColE}|}%
	{\gnumericPB{\raggedleft}\gnumbox[r]{{\color[rgb]{0.00,0.50,0.00} 3}}}
	&\multicolumn{1}{p{\gnumericColF}|}%
	{}
	&\multicolumn{1}{p{\gnumericColG}|}%
	{}
	&\multicolumn{1}{p{\gnumericColH}|}%
	{}
	&\multicolumn{1}{p{\gnumericColI}|}%
	{}
	&\multicolumn{1}{p{\gnumericColJ}|}%
	{}
	&\multicolumn{1}{p{\gnumericColK}|}%
	{}
	&\multicolumn{1}{p{\gnumericColL}|}%
	{}
	&\multicolumn{1}{p{\gnumericColM}|}%
	{}
	&\multicolumn{1}{p{\gnumericColN}|}%
	{}
	&\multicolumn{1}{p{\gnumericColO}|}%
	{}
	&\multicolumn{1}{p{\gnumericColP}|}%
	{}
	&\multicolumn{1}{p{\gnumericColQ}|}%
	{}
	&\multicolumn{1}{p{\gnumericColR}|}%
	{}
	&\multicolumn{1}{p{\gnumericColS}|}%
	{\gnumericPB{\raggedleft}\gnumbox[r]{{\color[rgb]{0.00,0.50,0.00} 1}}}
	&\multicolumn{1}{p{\gnumericColT}|}%
	{}
	&\multicolumn{1}{p{\gnumericColU}|}%
	{}
	&\multicolumn{1}{p{\gnumericColV}|}%
	{}
	&\multicolumn{1}{p{\gnumericColW}|}%
	{}
	&\multicolumn{1}{p{\gnumericColX}|}%
	{\gnumericPB{\raggedleft}\gnumbox[r]{{\color[rgb]{0.00,0.50,0.00} 6}}}
	&
\\
\hhline{~---------------------|~|--}
	 \gnumericPB{\raggedright}\gnumbox[l]{}
	&\multicolumn{1}{p{\gnumericColB}|}%
	{\gnumericPB{\centering}\gnumbox{{\color[rgb]{0.00,0.50,0.00} \textbf{\textit{Ist}}}}}
	&\multicolumn{1}{p{\gnumericColC}|}%
	{\gnumericPB{\raggedleft}\gnumbox[r]{{\color[rgb]{0.00,0.50,0.00} \textbf{\textit{2}}}}}
	&\multicolumn{1}{p{\gnumericColD}|}%
	{}
	&\multicolumn{1}{p{\gnumericColE}|}%
	{}
	&\multicolumn{1}{p{\gnumericColF}|}%
	{}
	&\multicolumn{1}{p{\gnumericColG}|}%
	{}
	&\multicolumn{1}{p{\gnumericColH}|}%
	{}
	&\multicolumn{1}{p{\gnumericColI}|}%
	{}
	&\multicolumn{1}{p{\gnumericColJ}|}%
	{}
	&\multicolumn{1}{p{\gnumericColK}|}%
	{}
	&\multicolumn{1}{p{\gnumericColL}|}%
	{}
	&\multicolumn{1}{p{\gnumericColM}|}%
	{}
	&\multicolumn{1}{p{\gnumericColN}|}%
	{}
	&\multicolumn{1}{p{\gnumericColO}|}%
	{}
	&\multicolumn{1}{p{\gnumericColP}|}%
	{}
	&\multicolumn{1}{p{\gnumericColQ}|}%
	{}
	&\multicolumn{1}{p{\gnumericColR}|}%
	{}
	&\multicolumn{1}{p{\gnumericColS}|}%
	{}
	&\multicolumn{1}{p{\gnumericColT}|}%
	{}
	&\multicolumn{1}{p{\gnumericColU}|}%
	{}
	&\multicolumn{1}{p{\gnumericColV}|}%
	{}
	&\multicolumn{1}{p{\gnumericColW}|}%
	{}
	&\multicolumn{1}{p{\gnumericColX}|}%
	{}
	&\multicolumn{1}{p{\gnumericColY}|}%
	{\gnumericPB{\raggedleft}\gnumbox[r]{\textbf{\textit{2}}}}
\\
\hhline{~---------------------|~|--|}
	 \gnumericPB{\raggedright}\gnumbox[l]{DRBD}
	&\multicolumn{1}{p{\gnumericColB}|}%
	{\gnumericPB{\centering}\gnumbox{{\color[rgb]{0.00,0.50,0.00} Soll}}}
	&\multicolumn{1}{p{\gnumericColC}|}%
	{}
	&\multicolumn{1}{p{\gnumericColD}|}%
	{}
	&\multicolumn{1}{p{\gnumericColE}|}%
	{}
	&\multicolumn{1}{p{\gnumericColF}|}%
	{\gnumericPB{\raggedleft}\gnumbox[r]{{\color[rgb]{0.00,0.50,0.00} 1}}}
	&\multicolumn{1}{p{\gnumericColG}|}%
	{\gnumericPB{\raggedleft}\gnumbox[r]{{\color[rgb]{0.00,0.50,0.00} 3}}}
	&\multicolumn{1}{p{\gnumericColH}|}%
	{\gnumericPB{\raggedleft}\gnumbox[r]{{\color[rgb]{0.00,0.50,0.00} 2}}}
	&\multicolumn{1}{p{\gnumericColI}|}%
	{}
	&\multicolumn{1}{p{\gnumericColJ}|}%
	{}
	&\multicolumn{1}{p{\gnumericColK}|}%
	{}
	&\multicolumn{1}{p{\gnumericColL}|}%
	{}
	&\multicolumn{1}{p{\gnumericColM}|}%
	{}
	&\multicolumn{1}{p{\gnumericColN}|}%
	{\gnumericPB{\raggedleft}\gnumbox[r]{{\color[rgb]{0.00,0.50,0.00} 1}}}
	&\multicolumn{1}{p{\gnumericColO}|}%
	{}
	&\multicolumn{1}{p{\gnumericColP}|}%
	{}
	&\multicolumn{1}{p{\gnumericColQ}|}%
	{}
	&\multicolumn{1}{p{\gnumericColR}|}%
	{}
	&\multicolumn{1}{p{\gnumericColS}|}%
	{}
	&\multicolumn{1}{p{\gnumericColT}|}%
	{}
	&\multicolumn{1}{p{\gnumericColU}|}%
	{}
	&\multicolumn{1}{p{\gnumericColV}|}%
	{}
	&\multicolumn{1}{p{\gnumericColW}|}%
	{}
	&\multicolumn{1}{p{\gnumericColX}|}%
	{\gnumericPB{\raggedleft}\gnumbox[r]{{\color[rgb]{0.00,0.50,0.00} 7}}}
	&
\\
\hhline{~---------------------|~|--}
	 
	&\multicolumn{1}{p{\gnumericColB}|}%
	{\gnumericPB{\centering}\gnumbox{{\color[rgb]{0.00,0.50,0.00} \textbf{\textit{Ist}}}}}
	&\multicolumn{1}{p{\gnumericColC}|}%
	{}
	&\multicolumn{1}{p{\gnumericColD}|}%
	{}
	&\multicolumn{1}{p{\gnumericColE}|}%
	{}
	&\multicolumn{1}{p{\gnumericColF}|}%
	{}
	&\multicolumn{1}{p{\gnumericColG}|}%
	{}
	&\multicolumn{1}{p{\gnumericColH}|}%
	{}
	&\multicolumn{1}{p{\gnumericColI}|}%
	{}
	&\multicolumn{1}{p{\gnumericColJ}|}%
	{}
	&\multicolumn{1}{p{\gnumericColK}|}%
	{}
	&\multicolumn{1}{p{\gnumericColL}|}%
	{}
	&\multicolumn{1}{p{\gnumericColM}|}%
	{}
	&\multicolumn{1}{p{\gnumericColN}|}%
	{}
	&\multicolumn{1}{p{\gnumericColO}|}%
	{}
	&\multicolumn{1}{p{\gnumericColP}|}%
	{}
	&\multicolumn{1}{p{\gnumericColQ}|}%
	{}
	&\multicolumn{1}{p{\gnumericColR}|}%
	{}
	&\multicolumn{1}{p{\gnumericColS}|}%
	{}
	&\multicolumn{1}{p{\gnumericColT}|}%
	{}
	&\multicolumn{1}{p{\gnumericColU}|}%
	{}
	&\multicolumn{1}{p{\gnumericColV}|}%
	{}
	&\multicolumn{1}{p{\gnumericColW}|}%
	{}
	&\multicolumn{1}{p{\gnumericColX}|}%
	{}
	&\multicolumn{1}{p{\gnumericColY}|}%
	{\gnumericPB{\raggedleft}\gnumbox[r]{\textbf{\textit{0}}}}
\\
\hhline{~---------------------|~|--|}
	 \gnumericPB{\raggedright}\gnumbox[l]{Pacemaker}
	&\multicolumn{1}{p{\gnumericColB}|}%
	{\gnumericPB{\centering}\gnumbox{{\color[rgb]{0.00,0.50,0.00} Soll}}}
	&\multicolumn{1}{p{\gnumericColC}|}%
	{}
	&\multicolumn{1}{p{\gnumericColD}|}%
	{}
	&\multicolumn{1}{p{\gnumericColE}|}%
	{}
	&\multicolumn{1}{p{\gnumericColF}|}%
	{}
	&\multicolumn{1}{p{\gnumericColG}|}%
	{}
	&\multicolumn{1}{p{\gnumericColH}|}%
	{}
	&\multicolumn{1}{p{\gnumericColI}|}%
	{\gnumericPB{\raggedleft}\gnumbox[r]{{\color[rgb]{0.00,0.50,0.00} 3}}}
	&\multicolumn{1}{p{\gnumericColJ}|}%
	{\gnumericPB{\raggedleft}\gnumbox[r]{{\color[rgb]{0.00,0.50,0.00} 3}}}
	&\multicolumn{1}{p{\gnumericColK}|}%
	{}
	&\multicolumn{1}{p{\gnumericColL}|}%
	{}
	&\multicolumn{1}{p{\gnumericColM}|}%
	{}
	&\multicolumn{1}{p{\gnumericColN}|}%
	{\gnumericPB{\raggedleft}\gnumbox[r]{{\color[rgb]{0.00,0.50,0.00} 1}}}
	&\multicolumn{1}{p{\gnumericColO}|}%
	{}
	&\multicolumn{1}{p{\gnumericColP}|}%
	{}
	&\multicolumn{1}{p{\gnumericColQ}|}%
	{}
	&\multicolumn{1}{p{\gnumericColR}|}%
	{}
	&\multicolumn{1}{p{\gnumericColS}|}%
	{}
	&\multicolumn{1}{p{\gnumericColT}|}%
	{}
	&\multicolumn{1}{p{\gnumericColU}|}%
	{}
	&\multicolumn{1}{p{\gnumericColV}|}%
	{}
	&\multicolumn{1}{p{\gnumericColW}|}%
	{}
	&\multicolumn{1}{p{\gnumericColX}|}%
	{\gnumericPB{\raggedleft}\gnumbox[r]{{\color[rgb]{0.00,0.50,0.00} 7}}}
	&
\\
\hhline{~---------------------|~|--}
	 \gnumericPB{\raggedright}\gnumbox[l]{}
	&\multicolumn{1}{p{\gnumericColB}|}%
	{\gnumericPB{\centering}\gnumbox{{\color[rgb]{0.00,0.50,0.00} \textbf{\textit{Ist}}}}}
	&\multicolumn{1}{p{\gnumericColC}|}%
	{}
	&\multicolumn{1}{p{\gnumericColD}|}%
	{}
	&\multicolumn{1}{p{\gnumericColE}|}%
	{}
	&\multicolumn{1}{p{\gnumericColF}|}%
	{}
	&\multicolumn{1}{p{\gnumericColG}|}%
	{}
	&\multicolumn{1}{p{\gnumericColH}|}%
	{}
	&\multicolumn{1}{p{\gnumericColI}|}%
	{}
	&\multicolumn{1}{p{\gnumericColJ}|}%
	{}
	&\multicolumn{1}{p{\gnumericColK}|}%
	{}
	&\multicolumn{1}{p{\gnumericColL}|}%
	{}
	&\multicolumn{1}{p{\gnumericColM}|}%
	{}
	&\multicolumn{1}{p{\gnumericColN}|}%
	{}
	&\multicolumn{1}{p{\gnumericColO}|}%
	{}
	&\multicolumn{1}{p{\gnumericColP}|}%
	{}
	&\multicolumn{1}{p{\gnumericColQ}|}%
	{}
	&\multicolumn{1}{p{\gnumericColR}|}%
	{}
	&\multicolumn{1}{p{\gnumericColS}|}%
	{}
	&\multicolumn{1}{p{\gnumericColT}|}%
	{}
	&\multicolumn{1}{p{\gnumericColU}|}%
	{}
	&\multicolumn{1}{p{\gnumericColV}|}%
	{}
	&\multicolumn{1}{p{\gnumericColW}|}%
	{}
	&\multicolumn{1}{p{\gnumericColX}|}%
	{}
	&\multicolumn{1}{p{\gnumericColY}|}%
	{\gnumericPB{\raggedleft}\gnumbox[r]{\textbf{\textit{0}}}}
\\
\hhline{~---------------------|~|--|}
	 \gnumericPB{\raggedright}\gnumbox[l]{RHEV}
	&\multicolumn{1}{p{\gnumericColB}|}%
	{\gnumericPB{\centering}\gnumbox{{\color[rgb]{0.00,0.50,0.00} Soll}}}
	&\multicolumn{1}{p{\gnumericColC}|}%
	{}
	&\multicolumn{1}{p{\gnumericColD}|}%
	{}
	&\multicolumn{1}{p{\gnumericColE}|}%
	{}
	&\multicolumn{1}{p{\gnumericColF}|}%
	{}
	&\multicolumn{1}{p{\gnumericColG}|}%
	{}
	&\multicolumn{1}{p{\gnumericColH}|}%
	{}
	&\multicolumn{1}{p{\gnumericColI}|}%
	{}
	&\multicolumn{1}{p{\gnumericColJ}|}%
	{}
	&\multicolumn{1}{p{\gnumericColK}|}%
	{\gnumericPB{\raggedleft}\gnumbox[r]{{\color[rgb]{0.00,0.50,0.00} 1}}}
	&\multicolumn{1}{p{\gnumericColL}|}%
	{\gnumericPB{\raggedleft}\gnumbox[r]{{\color[rgb]{0.00,0.50,0.00} 2}}}
	&\multicolumn{1}{p{\gnumericColM}|}%
	{\gnumericPB{\raggedleft}\gnumbox[r]{{\color[rgb]{0.00,0.50,0.00} 2}}}
	&\multicolumn{1}{p{\gnumericColN}|}%
	{\gnumericPB{\raggedleft}\gnumbox[r]{{\color[rgb]{0.00,0.50,0.00} 1}}}
	&\multicolumn{1}{p{\gnumericColO}|}%
	{}
	&\multicolumn{1}{p{\gnumericColP}|}%
	{}
	&\multicolumn{1}{p{\gnumericColQ}|}%
	{}
	&\multicolumn{1}{p{\gnumericColR}|}%
	{}
	&\multicolumn{1}{p{\gnumericColS}|}%
	{}
	&\multicolumn{1}{p{\gnumericColT}|}%
	{}
	&\multicolumn{1}{p{\gnumericColU}|}%
	{}
	&\multicolumn{1}{p{\gnumericColV}|}%
	{}
	&\multicolumn{1}{p{\gnumericColW}|}%
	{}
	&\multicolumn{1}{p{\gnumericColX}|}%
	{\gnumericPB{\raggedleft}\gnumbox[r]{{\color[rgb]{0.00,0.50,0.00} 6}}}
	&
\\
\hhline{~---------------------|~|--}
	 \gnumericPB{\raggedright}\gnumbox[l]{}
	&\multicolumn{1}{p{\gnumericColB}|}%
	{\gnumericPB{\centering}\gnumbox{{\color[rgb]{0.00,0.50,0.00} \textbf{\textit{Ist}}}}}
	&\multicolumn{1}{p{\gnumericColC}|}%
	{}
	&\multicolumn{1}{p{\gnumericColD}|}%
	{}
	&\multicolumn{1}{p{\gnumericColE}|}%
	{}
	&\multicolumn{1}{p{\gnumericColF}|}%
	{}
	&\multicolumn{1}{p{\gnumericColG}|}%
	{}
	&\multicolumn{1}{p{\gnumericColH}|}%
	{}
	&\multicolumn{1}{p{\gnumericColI}|}%
	{}
	&\multicolumn{1}{p{\gnumericColJ}|}%
	{}
	&\multicolumn{1}{p{\gnumericColK}|}%
	{}
	&\multicolumn{1}{p{\gnumericColL}|}%
	{}
	&\multicolumn{1}{p{\gnumericColM}|}%
	{}
	&\multicolumn{1}{p{\gnumericColN}|}%
	{}
	&\multicolumn{1}{p{\gnumericColO}|}%
	{}
	&\multicolumn{1}{p{\gnumericColP}|}%
	{}
	&\multicolumn{1}{p{\gnumericColQ}|}%
	{}
	&\multicolumn{1}{p{\gnumericColR}|}%
	{}
	&\multicolumn{1}{p{\gnumericColS}|}%
	{}
	&\multicolumn{1}{p{\gnumericColT}|}%
	{}
	&\multicolumn{1}{p{\gnumericColU}|}%
	{}
	&\multicolumn{1}{p{\gnumericColV}|}%
	{}
	&\multicolumn{1}{p{\gnumericColW}|}%
	{}
	&\multicolumn{1}{p{\gnumericColX}|}%
	{}
	&\multicolumn{1}{p{\gnumericColY}|}%
	{\gnumericPB{\raggedleft}\gnumbox[r]{\textbf{\textit{0}}}}
\\
\hhline{~---------------------|~|--|}
	 \gnumericPB{\raggedright}\gnumbox[l]{Testing}
	&\multicolumn{1}{p{\gnumericColB}|}%
	{\gnumericPB{\centering}\gnumbox{{\color[rgb]{0.00,0.50,0.00} Soll}}}
	&\multicolumn{1}{p{\gnumericColC}|}%
	{}
	&\multicolumn{1}{p{\gnumericColD}|}%
	{}
	&\multicolumn{1}{p{\gnumericColE}|}%
	{}
	&\multicolumn{1}{p{\gnumericColF}|}%
	{}
	&\multicolumn{1}{p{\gnumericColG}|}%
	{}
	&\multicolumn{1}{p{\gnumericColH}|}%
	{}
	&\multicolumn{1}{p{\gnumericColI}|}%
	{}
	&\multicolumn{1}{p{\gnumericColJ}|}%
	{}
	&\multicolumn{1}{p{\gnumericColK}|}%
	{}
	&\multicolumn{1}{p{\gnumericColL}|}%
	{}
	&\multicolumn{1}{p{\gnumericColM}|}%
	{}
	&\multicolumn{1}{p{\gnumericColN}|}%
	{}
	&\multicolumn{1}{p{\gnumericColO}|}%
	{}
	&\multicolumn{1}{p{\gnumericColP}|}%
	{}
	&\multicolumn{1}{p{\gnumericColQ}|}%
	{\gnumericPB{\raggedleft}\gnumbox[r]{{\color[rgb]{0.00,0.50,0.00} 3}}}
	&\multicolumn{1}{p{\gnumericColR}|}%
	{\gnumericPB{\raggedleft}\gnumbox[r]{{\color[rgb]{0.00,0.50,0.00} 3}}}
	&\multicolumn{1}{p{\gnumericColS}|}%
	{\gnumericPB{\raggedleft}\gnumbox[r]{{\color[rgb]{0.00,0.50,0.00} 2}}}
	&\multicolumn{1}{p{\gnumericColT}|}%
	{}
	&\multicolumn{1}{p{\gnumericColU}|}%
	{}
	&\multicolumn{1}{p{\gnumericColV}|}%
	{}
	&\multicolumn{1}{p{\gnumericColW}|}%
	{}
	&\multicolumn{1}{p{\gnumericColX}|}%
	{\gnumericPB{\raggedleft}\gnumbox[r]{{\color[rgb]{0.00,0.50,0.00} 8}}}
	&
\\
\hhline{~---------------------|~|--}
	 \gnumericPB{\raggedright}\gnumbox[l]{}
	&\multicolumn{1}{p{\gnumericColB}|}%
	{\gnumericPB{\centering}\gnumbox{{\color[rgb]{0.00,0.50,0.00} \textbf{\textit{Ist}}}}}
	&\multicolumn{1}{p{\gnumericColC}|}%
	{}
	&\multicolumn{1}{p{\gnumericColD}|}%
	{}
	&\multicolumn{1}{p{\gnumericColE}|}%
	{}
	&\multicolumn{1}{p{\gnumericColF}|}%
	{}
	&\multicolumn{1}{p{\gnumericColG}|}%
	{}
	&\multicolumn{1}{p{\gnumericColH}|}%
	{}
	&\multicolumn{1}{p{\gnumericColI}|}%
	{}
	&\multicolumn{1}{p{\gnumericColJ}|}%
	{}
	&\multicolumn{1}{p{\gnumericColK}|}%
	{}
	&\multicolumn{1}{p{\gnumericColL}|}%
	{}
	&\multicolumn{1}{p{\gnumericColM}|}%
	{}
	&\multicolumn{1}{p{\gnumericColN}|}%
	{}
	&\multicolumn{1}{p{\gnumericColO}|}%
	{}
	&\multicolumn{1}{p{\gnumericColP}|}%
	{}
	&\multicolumn{1}{p{\gnumericColQ}|}%
	{}
	&\multicolumn{1}{p{\gnumericColR}|}%
	{}
	&\multicolumn{1}{p{\gnumericColS}|}%
	{}
	&\multicolumn{1}{p{\gnumericColT}|}%
	{}
	&\multicolumn{1}{p{\gnumericColU}|}%
	{}
	&\multicolumn{1}{p{\gnumericColV}|}%
	{}
	&\multicolumn{1}{p{\gnumericColW}|}%
	{}
	&\multicolumn{1}{p{\gnumericColX}|}%
	{}
	&\multicolumn{1}{p{\gnumericColY}|}%
	{\gnumericPB{\raggedleft}\gnumbox[r]{\textbf{\textit{0}}}}
\\
\hhline{~---------------------|~|--|}
	 \gnumericPB{\raggedright}\gnumbox[l]{Migration der Server}
	&\multicolumn{1}{p{\gnumericColB}|}%
	{\gnumericPB{\centering}\gnumbox{{\color[rgb]{0.00,0.50,0.00} Soll}}}
	&\multicolumn{1}{p{\gnumericColC}|}%
	{}
	&\multicolumn{1}{p{\gnumericColD}|}%
	{}
	&\multicolumn{1}{p{\gnumericColE}|}%
	{}
	&\multicolumn{1}{p{\gnumericColF}|}%
	{}
	&\multicolumn{1}{p{\gnumericColG}|}%
	{}
	&\multicolumn{1}{p{\gnumericColH}|}%
	{}
	&\multicolumn{1}{p{\gnumericColI}|}%
	{}
	&\multicolumn{1}{p{\gnumericColJ}|}%
	{}
	&\multicolumn{1}{p{\gnumericColK}|}%
	{}
	&\multicolumn{1}{p{\gnumericColL}|}%
	{}
	&\multicolumn{1}{p{\gnumericColM}|}%
	{}
	&\multicolumn{1}{p{\gnumericColN}|}%
	{}
	&\multicolumn{1}{p{\gnumericColO}|}%
	{}
	&\multicolumn{1}{p{\gnumericColP}|}%
	{}
	&\multicolumn{1}{p{\gnumericColQ}|}%
	{}
	&\multicolumn{1}{p{\gnumericColR}|}%
	{}
	&\multicolumn{1}{p{\gnumericColS}|}%
	{}
	&\multicolumn{1}{p{\gnumericColT}|}%
	{}
	&\multicolumn{1}{p{\gnumericColU}|}%
	{}
	&\multicolumn{1}{p{\gnumericColV}|}%
	{}
	&\multicolumn{1}{p{\gnumericColW}|}%
	{}
	&\multicolumn{1}{p{\gnumericColX}|}%
	{\gnumericPB{\raggedleft}\gnumbox[r]{{\color[rgb]{0.00,0.50,0.00} 0}}}
	&
\\
\hhline{~---------------------|~|--}
	 \gnumericPB{\raggedright}\gnumbox[l]{}
	&\multicolumn{1}{p{\gnumericColB}|}%
	{\gnumericPB{\centering}\gnumbox{{\color[rgb]{0.00,0.50,0.00} \textbf{\textit{Ist}}}}}
	&\multicolumn{1}{p{\gnumericColC}|}%
	{}
	&\multicolumn{1}{p{\gnumericColD}|}%
	{}
	&\multicolumn{1}{p{\gnumericColE}|}%
	{}
	&\multicolumn{1}{p{\gnumericColF}|}%
	{}
	&\multicolumn{1}{p{\gnumericColG}|}%
	{}
	&\multicolumn{1}{p{\gnumericColH}|}%
	{}
	&\multicolumn{1}{p{\gnumericColI}|}%
	{}
	&\multicolumn{1}{p{\gnumericColJ}|}%
	{}
	&\multicolumn{1}{p{\gnumericColK}|}%
	{}
	&\multicolumn{1}{p{\gnumericColL}|}%
	{}
	&\multicolumn{1}{p{\gnumericColM}|}%
	{}
	&\multicolumn{1}{p{\gnumericColN}|}%
	{}
	&\multicolumn{1}{p{\gnumericColO}|}%
	{}
	&\multicolumn{1}{p{\gnumericColP}|}%
	{}
	&\multicolumn{1}{p{\gnumericColQ}|}%
	{}
	&\multicolumn{1}{p{\gnumericColR}|}%
	{}
	&\multicolumn{1}{p{\gnumericColS}|}%
	{}
	&\multicolumn{1}{p{\gnumericColT}|}%
	{}
	&\multicolumn{1}{p{\gnumericColU}|}%
	{}
	&\multicolumn{1}{p{\gnumericColV}|}%
	{}
	&\multicolumn{1}{p{\gnumericColW}|}%
	{}
	&\multicolumn{1}{p{\gnumericColX}|}%
	{}
	&\multicolumn{1}{p{\gnumericColY}|}%
	{\gnumericPB{\raggedleft}\gnumbox[r]{\textbf{\textit{0}}}}
\\
\hhline{~---------------------|~|--|}
	 \gnumericPB{\raggedright}\gnumbox[l]{Testing}
	&\multicolumn{1}{p{\gnumericColB}|}%
	{\gnumericPB{\centering}\gnumbox{{\color[rgb]{0.00,0.50,0.00} Soll}}}
	&\multicolumn{1}{p{\gnumericColC}|}%
	{}
	&\multicolumn{1}{p{\gnumericColD}|}%
	{}
	&\multicolumn{1}{p{\gnumericColE}|}%
	{}
	&\multicolumn{1}{p{\gnumericColF}|}%
	{}
	&\multicolumn{1}{p{\gnumericColG}|}%
	{}
	&\multicolumn{1}{p{\gnumericColH}|}%
	{}
	&\multicolumn{1}{p{\gnumericColI}|}%
	{}
	&\multicolumn{1}{p{\gnumericColJ}|}%
	{}
	&\multicolumn{1}{p{\gnumericColK}|}%
	{}
	&\multicolumn{1}{p{\gnumericColL}|}%
	{}
	&\multicolumn{1}{p{\gnumericColM}|}%
	{}
	&\multicolumn{1}{p{\gnumericColN}|}%
	{}
	&\multicolumn{1}{p{\gnumericColO}|}%
	{}
	&\multicolumn{1}{p{\gnumericColP}|}%
	{}
	&\multicolumn{1}{p{\gnumericColQ}|}%
	{}
	&\multicolumn{1}{p{\gnumericColR}|}%
	{}
	&\multicolumn{1}{p{\gnumericColS}|}%
	{}
	&\multicolumn{1}{p{\gnumericColT}|}%
	{}
	&\multicolumn{1}{p{\gnumericColU}|}%
	{}
	&\multicolumn{1}{p{\gnumericColV}|}%
	{}
	&\multicolumn{1}{p{\gnumericColW}|}%
	{}
	&\multicolumn{1}{p{\gnumericColX}|}%
	{\gnumericPB{\raggedleft}\gnumbox[r]{{\color[rgb]{0.00,0.50,0.00} 0}}}
	&
\\
\hhline{~---------------------|~|--}
	 \gnumericPB{\raggedright}\gnumbox[l]{}
	&\multicolumn{1}{p{\gnumericColB}|}%
	{\gnumericPB{\centering}\gnumbox{{\color[rgb]{0.00,0.50,0.00} \textbf{\textit{Ist}}}}}
	&\multicolumn{1}{p{\gnumericColC}|}%
	{}
	&\multicolumn{1}{p{\gnumericColD}|}%
	{}
	&\multicolumn{1}{p{\gnumericColE}|}%
	{}
	&\multicolumn{1}{p{\gnumericColF}|}%
	{}
	&\multicolumn{1}{p{\gnumericColG}|}%
	{}
	&\multicolumn{1}{p{\gnumericColH}|}%
	{}
	&\multicolumn{1}{p{\gnumericColI}|}%
	{}
	&\multicolumn{1}{p{\gnumericColJ}|}%
	{}
	&\multicolumn{1}{p{\gnumericColK}|}%
	{}
	&\multicolumn{1}{p{\gnumericColL}|}%
	{}
	&\multicolumn{1}{p{\gnumericColM}|}%
	{}
	&\multicolumn{1}{p{\gnumericColN}|}%
	{}
	&\multicolumn{1}{p{\gnumericColO}|}%
	{}
	&\multicolumn{1}{p{\gnumericColP}|}%
	{}
	&\multicolumn{1}{p{\gnumericColQ}|}%
	{}
	&\multicolumn{1}{p{\gnumericColR}|}%
	{}
	&\multicolumn{1}{p{\gnumericColS}|}%
	{}
	&\multicolumn{1}{p{\gnumericColT}|}%
	{}
	&\multicolumn{1}{p{\gnumericColU}|}%
	{}
	&\multicolumn{1}{p{\gnumericColV}|}%
	{}
	&\multicolumn{1}{p{\gnumericColW}|}%
	{}
	&\multicolumn{1}{p{\gnumericColX}|}%
	{}
	&\multicolumn{1}{p{\gnumericColY}|}%
	{\gnumericPB{\raggedleft}\gnumbox[r]{\textbf{\textit{0}}}}
\\
\hhline{~---------------------|~|--|}
	 
	&\multicolumn{1}{p{\gnumericColB}|}%
	{}
	&\multicolumn{1}{p{\gnumericColC}|}%
	{}
	&\multicolumn{1}{p{\gnumericColD}|}%
	{}
	&\multicolumn{1}{p{\gnumericColE}|}%
	{}
	&\multicolumn{1}{p{\gnumericColF}|}%
	{}
	&\multicolumn{1}{p{\gnumericColG}|}%
	{}
	&\multicolumn{1}{p{\gnumericColH}|}%
	{}
	&\multicolumn{1}{p{\gnumericColI}|}%
	{}
	&\multicolumn{1}{p{\gnumericColJ}|}%
	{}
	&\multicolumn{1}{p{\gnumericColK}|}%
	{}
	&\multicolumn{1}{p{\gnumericColL}|}%
	{}
	&\multicolumn{1}{p{\gnumericColM}|}%
	{}
	&\multicolumn{1}{p{\gnumericColN}|}%
	{}
	&\multicolumn{1}{p{\gnumericColO}|}%
	{}
	&\multicolumn{1}{p{\gnumericColP}|}%
	{}
	&\multicolumn{1}{p{\gnumericColQ}|}%
	{}
	&\multicolumn{1}{p{\gnumericColR}|}%
	{}
	&\multicolumn{1}{p{\gnumericColS}|}%
	{}
	&\multicolumn{1}{p{\gnumericColT}|}%
	{}
	&\multicolumn{1}{p{\gnumericColU}|}%
	{}
	&\multicolumn{1}{p{\gnumericColV}|}%
	{}
	&\multicolumn{1}{p{\gnumericColW}|}%
	{}
	&\multicolumn{1}{p{\gnumericColX}|}%
	{}
	&
\\
\hhline{~---------------------|~|-|~}
	 
	&\multicolumn{1}{p{\gnumericColB}|}%
	{\gnumericPB{\centering}\gnumbox{{\color[rgb]{0.00,0.50,0.00} Soll / Halbtag}}}
	&\multicolumn{1}{p{\gnumericColC}|}%
	{\gnumericPB{\raggedleft}\gnumbox[r]{{\color[rgb]{0.00,0.50,0.00} 4}}}
	&\multicolumn{1}{p{\gnumericColD}|}%
	{\gnumericPB{\raggedleft}\gnumbox[r]{{\color[rgb]{0.00,0.50,0.00} 4}}}
	&\multicolumn{1}{p{\gnumericColE}|}%
	{\gnumericPB{\raggedleft}\gnumbox[r]{{\color[rgb]{0.00,0.50,0.00} 4}}}
	&\multicolumn{1}{p{\gnumericColF}|}%
	{\gnumericPB{\raggedleft}\gnumbox[r]{{\color[rgb]{0.00,0.50,0.00} 4}}}
	&\multicolumn{1}{p{\gnumericColG}|}%
	{\gnumericPB{\raggedleft}\gnumbox[r]{{\color[rgb]{0.00,0.50,0.00} 4}}}
	&\multicolumn{1}{p{\gnumericColH}|}%
	{\gnumericPB{\raggedleft}\gnumbox[r]{{\color[rgb]{0.00,0.50,0.00} 4}}}
	&\multicolumn{1}{p{\gnumericColI}|}%
	{\gnumericPB{\raggedleft}\gnumbox[r]{{\color[rgb]{0.00,0.50,0.00} 4}}}
	&\multicolumn{1}{p{\gnumericColJ}|}%
	{\gnumericPB{\raggedleft}\gnumbox[r]{{\color[rgb]{0.00,0.50,0.00} 4}}}
	&\multicolumn{1}{p{\gnumericColK}|}%
	{\gnumericPB{\raggedleft}\gnumbox[r]{{\color[rgb]{0.00,0.50,0.00} 4}}}
	&\multicolumn{1}{p{\gnumericColL}|}%
	{\gnumericPB{\raggedleft}\gnumbox[r]{{\color[rgb]{0.00,0.50,0.00} 4}}}
	&\multicolumn{1}{p{\gnumericColM}|}%
	{\gnumericPB{\raggedleft}\gnumbox[r]{{\color[rgb]{0.00,0.50,0.00} 4}}}
	&\multicolumn{1}{p{\gnumericColN}|}%
	{\gnumericPB{\raggedleft}\gnumbox[r]{{\color[rgb]{0.00,0.50,0.00} 4}}}
	&\multicolumn{1}{p{\gnumericColO}|}%
	{\gnumericPB{\raggedleft}\gnumbox[r]{{\color[rgb]{0.00,0.50,0.00} 4}}}
	&\multicolumn{1}{p{\gnumericColP}|}%
	{\gnumericPB{\raggedleft}\gnumbox[r]{{\color[rgb]{0.00,0.50,0.00} 4}}}
	&\multicolumn{1}{p{\gnumericColQ}|}%
	{\gnumericPB{\raggedleft}\gnumbox[r]{{\color[rgb]{0.00,0.50,0.00} 4}}}
	&\multicolumn{1}{p{\gnumericColR}|}%
	{\gnumericPB{\raggedleft}\gnumbox[r]{{\color[rgb]{0.00,0.50,0.00} 4}}}
	&\multicolumn{1}{p{\gnumericColS}|}%
	{\gnumericPB{\raggedleft}\gnumbox[r]{{\color[rgb]{0.00,0.50,0.00} 4}}}
	&\multicolumn{1}{p{\gnumericColT}|}%
	{\gnumericPB{\raggedleft}\gnumbox[r]{{\color[rgb]{0.00,0.50,0.00} 4}}}
	&\multicolumn{1}{p{\gnumericColU}|}%
	{\gnumericPB{\raggedleft}\gnumbox[r]{{\color[rgb]{0.00,0.50,0.00} 4}}}
	&\multicolumn{1}{p{\gnumericColV}|}%
	{\gnumericPB{\raggedleft}\gnumbox[r]{{\color[rgb]{0.00,0.50,0.00} 4}}}
	&\multicolumn{1}{p{\gnumericColW}|}%
	{}
	&\multicolumn{1}{p{\gnumericColX}|}%
	{\gnumericPB{\raggedleft}\gnumbox[r]{{\color[rgb]{0.00,0.50,0.00} 80}}}
	&
\\
\hhline{~---------------------|~|--}
	 
	&\multicolumn{1}{p{\gnumericColB}|}%
	{\gnumericPB{\centering}\gnumbox{{\color[rgb]{0.00,0.50,0.00} Ist / Halbtag}}}
	&\multicolumn{1}{p{\gnumericColC}|}%
	{\gnumericPB{\raggedleft}\gnumbox[r]{{\color[rgb]{0.00,0.50,0.00} 4}}}
	&\multicolumn{1}{p{\gnumericColD}|}%
	{\gnumericPB{\raggedleft}\gnumbox[r]{{\color[rgb]{0.00,0.50,0.00} 0}}}
	&\multicolumn{1}{p{\gnumericColE}|}%
	{\gnumericPB{\raggedleft}\gnumbox[r]{{\color[rgb]{0.00,0.50,0.00} 0}}}
	&\multicolumn{1}{p{\gnumericColF}|}%
	{\gnumericPB{\raggedleft}\gnumbox[r]{{\color[rgb]{0.00,0.50,0.00} 0}}}
	&\multicolumn{1}{p{\gnumericColG}|}%
	{\gnumericPB{\raggedleft}\gnumbox[r]{{\color[rgb]{0.00,0.50,0.00} 0}}}
	&\multicolumn{1}{p{\gnumericColH}|}%
	{\gnumericPB{\raggedleft}\gnumbox[r]{{\color[rgb]{0.00,0.50,0.00} 0}}}
	&\multicolumn{1}{p{\gnumericColI}|}%
	{\gnumericPB{\raggedleft}\gnumbox[r]{{\color[rgb]{0.00,0.50,0.00} 0}}}
	&\multicolumn{1}{p{\gnumericColJ}|}%
	{\gnumericPB{\raggedleft}\gnumbox[r]{{\color[rgb]{0.00,0.50,0.00} 0}}}
	&\multicolumn{1}{p{\gnumericColK}|}%
	{\gnumericPB{\raggedleft}\gnumbox[r]{{\color[rgb]{0.00,0.50,0.00} 0}}}
	&\multicolumn{1}{p{\gnumericColL}|}%
	{\gnumericPB{\raggedleft}\gnumbox[r]{{\color[rgb]{0.00,0.50,0.00} 0}}}
	&\multicolumn{1}{p{\gnumericColM}|}%
	{\gnumericPB{\raggedleft}\gnumbox[r]{{\color[rgb]{0.00,0.50,0.00} 0}}}
	&\multicolumn{1}{p{\gnumericColN}|}%
	{\gnumericPB{\raggedleft}\gnumbox[r]{{\color[rgb]{0.00,0.50,0.00} 0}}}
	&\multicolumn{1}{p{\gnumericColO}|}%
	{\gnumericPB{\raggedleft}\gnumbox[r]{{\color[rgb]{0.00,0.50,0.00} 0}}}
	&\multicolumn{1}{p{\gnumericColP}|}%
	{\gnumericPB{\raggedleft}\gnumbox[r]{{\color[rgb]{0.00,0.50,0.00} 0}}}
	&\multicolumn{1}{p{\gnumericColQ}|}%
	{\gnumericPB{\raggedleft}\gnumbox[r]{{\color[rgb]{0.00,0.50,0.00} 0}}}
	&\multicolumn{1}{p{\gnumericColR}|}%
	{\gnumericPB{\raggedleft}\gnumbox[r]{{\color[rgb]{0.00,0.50,0.00} 0}}}
	&\multicolumn{1}{p{\gnumericColS}|}%
	{\gnumericPB{\raggedleft}\gnumbox[r]{{\color[rgb]{0.00,0.50,0.00} 0}}}
	&\multicolumn{1}{p{\gnumericColT}|}%
	{\gnumericPB{\raggedleft}\gnumbox[r]{{\color[rgb]{0.00,0.50,0.00} 0}}}
	&\multicolumn{1}{p{\gnumericColU}|}%
	{\gnumericPB{\raggedleft}\gnumbox[r]{{\color[rgb]{0.00,0.50,0.00} 0}}}
	&\multicolumn{1}{p{\gnumericColV}|}%
	{\gnumericPB{\raggedleft}\gnumbox[r]{{\color[rgb]{0.00,0.50,0.00} 0}}}
	&\multicolumn{1}{p{\gnumericColW}|}%
	{}
	&\multicolumn{1}{p{\gnumericColX}|}%
	{}
	&\multicolumn{1}{p{\gnumericColY}|}%
	{\gnumericPB{\raggedleft}\gnumbox[r]{4}}
\\
\hhline{~-|-|-|-|-|-|-|-|-|-|-|-|-|-|-|-|-|-|-|-|-|~|-|-|}
\end{longtable}

\ifthenelse{\isundefined{\languageshorthands}}{}{\languageshorthands{\languagename}}
\gnumericTableEnd

\input{part1/zeitplan}

\chapter{Arbeitsjournal}
Die Festlegungen dieses Dokuments gelten im Projekt .\\
Gemäss Art. 5 Absatz 2 der Wegleitung über die individuelle praktische Arbeit (IPA) an Lehrabschlussprüfungen des BBT vom 27. August 2001 gilt:\\
„Die zu prüfende Person führt ein Arbeitsjournal. Sie dokumentiert darin täglich das Vorgehen, den Stand der Prüfungsarbeit, sämtliche fremde Hilfestellungen und besondere Vorkommnisse wie z.B. Änderungen der Aufgabenstellung, Arbeitsunterbrüche, organisatorische Probleme, Abweichungen von der Soll-Planung.“\\
Das Arbeitsjournal zur IPA ist zwingend zu führen und den Experten und Fachvorgesetzten vorzulegen. Das Arbeitsjournal ist täglich sinngemäss und korrekt auszufüllen.\\
Das Arbeitsjournal dient der Nachvollziehbarkeit der von den Lernenden ausgeführten Arbeiten und wird als Teil der IPA in die Bewertung mit einbezogen.
\newpage
\input{journal/tag1.tex}
\input{journal/tag2.tex}
\input{journal/tag3.tex}
\input{journal/tag4.tex}
\input{journal/tag5.tex}
\input{journal/tag6.tex}
\input{journal/tag7.tex}
\input{journal/tag8.tex}
\section{Erster Tag: Dienstag 11. Februar 2014}

\begin{table}[htb]
	\begin{tabularx}{\textwidth}{ bss }
		\hline
		\textbf{Tätigkeiten} &  \textbf{Aufwand geplant (Std)} & \textbf{Aufwand effektiv (Std)}\\ \hline
		\textbf{Tagesplan / Daily Scrum}  &  1 & 1 \\ \hline
		\textbf{Recherche, Vorbereitungen:}  & 2h &2h \\ \hline
		\textbf{Dokumentation Projekt:} Erstellen des Zeitplans &  3h &3.5h \\ \hline
		\textbf{Dokumentation Projekt:} Eröffnen der Dokumentation &  2h &1.5h \\ \hline
	\end{tabularx}
\end{table}
\paragraph{Tagesablauf}

\paragraph{Hilfestellungen}    

\paragraph{Reflexion} 

\paragraph{Nächste Schritte} 
\begin{itemize}
	\item 
\end{itemize}
\newpage
\input{journal/tag10.tex}


\chapter{Abschlussbericht}
\section{Vergleich Ist/Soll} 

\section{Realisierungsbericht}

\section{Testbericht}

\section{Fazit zum Projekt}

\section{Persönliches Fazit}


\chapter{Unterschriften Teil I}
Die lernende Person bestätigt mit ihrer Unterschrift diese IPA aus Eigenleistung erbracht und nach den Vorgaben der Prüfungskommission Informatik Kanton Bern erstellt zu haben. Die Angaben im Arbeitsjournal entsprechen dem geleisteten Arbeitsaufwand.\\

\begin{table}
\begin{tabularx}{\textwidth}{ |X|X|X| }
\hline
Datum & Name / OE & Unterschrift \\ \hline
 & Name Vorname (Lernender) & \\[5ex] \hline
 & Name Vorname (Fachvorgesetzter) & \\[5ex] \hline
\end{tabularx}
\caption{Unterschriften Teil I}
\end{table}

\part{Projektdokumentation}
\chapter{Projektumriss}
\section{Projektmethode}

\section{Projektsprache}

\section{Vorgehensweise}

\section{Verwendete Mittel}

\section{System Übersicht}

\section{Netzwerk und Dienste}

\section{Filesystem}

\subsection{Überlegungen}


\chapter{Step-By-Step Anleitung}
\section{Einleitung}


\subsection{Ziel dieser Anleitung}

%\subsubsection{Einschränkungen}

\subsection{Voraussetzungen}

\section{Software 1}
\subsection{Installation}

\subsection{Konfiguration}

\section{KVM Maschine definieren}
Auf beiden Server muss eine virtuelle Maschine erstellt werden...
\section{Software 2}
\subsection{Installation}

\subsection{Konfiguration}

\section{Testen}


\chapter{Tests}
\section{Rahmenbedingungen für das Testing}

\section{Testszenario}
...
Dies gibt die folgenden Testfälle:
\subsection{Test 1}
\subsubsection{Methode}

\subsubsection{Erwartete Resultate}

\subsubsection{Tatsächliches Resultat}

\subsection{Test 2}
\subsubsection{Methode}

\subsubsection{Erwartete Resultate}

\subsubsection{Tatsächliches Resultat}

\subsection{Diverse kleinere Tests}
\subsubsection{Kleiner Test 1}
\section{Überlegungen}

%\chapter{Voranalyse}
%Die Voranalyse ist ein Klärungsprozess, der mit vertretbarem Aufwand eine Entscheidung über die grundsätzliche Art der Systemrealisierung herbeiführt.\\
Erstellung und Beurteilung der Situationsanalyse sowie Überprüfung der Zielesetzungen, der Problemstellung und des Untersuchungsbereichs. \\
Erarbeitung von Lösungsvorschlägen und Abschätzung ihrer voraussichtlichen (Wirtschaft-lichkeit) und Realisierbarkeit. Eine sinnvolle Risikoanalyse für das Projekt, welche Risiken eintreten könnten während -und nach dem Projekt 

%\chapter{Konzept}
%In der Konzepterarbeitung werden die Grundlagen für die Realisierung und Einführung eines Informatiksystems entwickelt.
Das Konzept wird schrittweise mit folgenden Schritten entwickelt.
Es ist wichtig, die Ergebnisse so weit zu deklarieren, dass damit die Systemarchitektur bestimmt werden kann. Abgestimmt mit der schrittweisen Entwicklung des Konzepts werden die Fertigprodukte ? Sachmittel evaluiert.

\section{Konzept entwickeln}
Systemanforderungen
Systemarchitektur
Materialbeschaffung
Systemintegrationsplan
Einführungskonzept
Datenmigration
Ausbildungskonzept
(Wirtschaftlichkeit)


\section{Fertigprodukte evaluieren (wenn keine Konzept Phase)}
Plichtenheft
Fertigproduktevaluation
Evaluationsbewertung

\section{Sachmittel evaluieren (wenn keine Konzept Phase)}
Sachmittelbedarf
Sachmittelevaluation
(Wirtschaftlichkeit)

\section{Schutzmassnahmen erarbeiten}
ISDS-Konzept

\section{Testkonzept}

%\chapter{Realisierung}
%\input{part2/realisierung}

%\chapter{Einführung / Abschluss}
%\input{part2/abschluss}

\chapter{Unterschriften Teil II}
Die lernende Person bestätigt mit ihrer Unterschrift diese IPA aus Eigenleistung erbracht und nach den Vorgaben der Prüfungskommission Informatik Kanton Bern erstellt zu haben. Die Angaben im Arbeitsjournal entsprechen dem geleisteten Arbeitsaufwand.\\

\begin{table}
\begin{tabularx}{\textwidth}{ |X|X|X| }
\hline
Datum & Name / OE & Unterschrift \\ \hline
 & Vorname Name (Lernender) & \\[5ex] \hline
 & Vorname Name (Fachvorgesetzter) & \\[5ex] \hline
\end{tabularx}
\caption{Unterschriften Teil II}
\end{table}

\part{Anhang}                                                                                                                                                                                                                                       
\section{Glossar}
\begin{tabularx}{\textwidth}{s|b}
\textbf{Begriff} & \textbf{Bedeutung} \\
KVM & Kernel-based Virtual Machine: Eine Virtualisierungsmöglichkeit unter Linux \\
\end{tabularx}
\pagebreak
\section{Verwendete Dateien}
\input{anhang/anhang.tex}
\end{document}
