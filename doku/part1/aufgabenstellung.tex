\section{Ausgangslage}
 
\section{Auftragsformulierung}
  
\section{Mittel und Methoden}
\begin{itemize}
	\item 
\end{itemize}
\section{Projektorganisation}
\textbf{Projektausschuss:} \\
Fachvorgesetzter: \\
Hauptexperte:  \\
Zweitexperte:  \\
\textbf{Customer:}  \\
\textbf{Product Owner:} \\
\textbf{Scrum Master:} \\
\textbf{Entwicklungsteam:} \\
\pagebreak
\section{Projektrollen}
Die verschiedenen Projektrollen nach Scrum: \\
\begin{table}[h]
\begin{tabularx}{\textwidth}{ |X|X| }
\hline
Customer & Auftraggeber, dem das Produkt nach Abschluss zur Verfügung gestellt wird \\ \hline
Product Owner & Stellt die strategischen Anforderungen. Festlegung der Ziele, Abnahme der Leistung.\\ \hline
Scrum Master & Verantwortlich für das Gelingen von Scrum. Kümmert sich um Hindernisse und sorgt für ein angenehmes Arbeitsklima. \\ \hline
Entwicklungsteam & Für die Lieferung der Produktfunktionalitäten zuständig. \\ \hline
\end{tabularx}
\caption{Projektrollenbeschrieb}
\end{table}